<k Erster Teil.>


Hans Jung ließ die Ruder aus der Hand gleiten,
daß sie lose auf dem Wasser schwammen, das
Boot blieb mitten im See liegen und schwankte langsam
hin und her. Die kleinen Wellen schlugen leis an
die Planken.

„Es ist schöner, sich treiben zu lassen,“ sagte
Frau Marie, die Hans Jung gegenüber auf der Bank
am Kiel saß und die Steuerschnüre in den Händen
hielt.

„Es wird bald regnen,“ sagte Hans Jung. Die
Uferberge waren in dicke Wolken gehüllt, und aus
allen Tälern kam der Nebel schwarz und schwer.
Manchmal fiel ein Tropfen in das Wasser, das gab
langsame, weite Kringel.

Aus dem breiten Bauernhaus, das einsam auf
der Ostseite des langen, schmalen Sees lag, war ein
Mädchen getreten. „Thomas!“[|,] rief sie über das
Wasser und hielt die Hände an den Mund. „Thomas!“
[s 7]
– Sie bekam Antwort aus einem der letzten Häuser
des Dorfes, die dicht am Ufer lagen. Ein junger
Bursch sah aus dem Fenster. „Wir kommen zum
Heimgarten heut!“[|,] rief das Mädchen. Sie bog sich
weit vor und schrie, daß sich das Mieder nur so um
ihre Brust spannte. Das Boot lag gerade in der
Mitte zwischen den beiden. „Heimgarten!“[|,] rief Hans
Jung dem Burschen zu. Er verstand ihn und juchzte
einen langen Schrei zu dem Mädchen hinüber, ehe
er aus dem Fenster verschwand. Sie antwortete mit
einem kleinen, weichen Jodler und ging ins Haus.
Jetzt erst kam der Juchzer aus den Bergen zurück,
ganz schwach. Und kaum vernehmlich hinterher der
kleine Jodler. Das Boot lag wieder einsam zwischen
den einzelnen leisen Kringeln.

Frau Marie hob den Kopf ein wenig. „Was
ist das – Heim­garten?“[|,] fragte sie.

„Da sitzen sie zusammen,“ sagte Hans Jung,
„im Winter um den Ofen und im Sommer im Garten,
zwischen den blühenden Büschen – –“, er unter­
brach sich selbst, ein wenig ärgerlich über seine
letzten Worte. „Sie sitzen zusammen und schwätzen,“
sagte er.

Frau Marie schwieg. Nebel kam und umspann
das Boot. Die Kringel waren nicht mehr so klein
und folgten einander schneller.

[s 8]
Hans Jung griff nach den Rudern. „Wir müssen
heim jetzt, sonst werden Sie naß,“ sagte er.

Frau Marie hatte den Kopf gewandt und sah
auf das Wasser hinaus. Jetzt, wo er nicht ihren
Augen begegnen mußte, konnte er sie wieder klar
und ohne Schleier sehen, das Bild, das ihm auf seiner
weiten Reise nur stückweise erschienen war – –
einmal der Mund, wie er leis lächelte und einmal
die Hand, wie sie sich hob.

Noch immer sah sie wie ein Mädchen aus.

Er bog sich weit in den Rudern zurück und sah
sie an, bis sie durch das kleine Schilf vor dem Boots­
haus rauschten.

„Ist es schon aus?“[|,] fragte Frau Marie, als er ihr
aus dem Boot half. Sie gingen langsam die nasse
Dorfstraße hinauf. Hie und da begegneten ihnen
Sommerfrischler, die ins Hotel zum Abendessen
wanderten. Die Damen machten spitze Schrittchen
über die vielen Pfützen und hoben die Röcke, daß
man unter dem bunten Bauerntuch die städtischen
Stiefelschäfte und Strümpfe zu sehen bekam. Die
Herren halfen ihnen und schauten auf ihre Füße.
Alle machten verschlafene Gesichter und schimpften
über den Regen. Manche grüßten Frau Marie.

„Essen Sie heute abend bei mir,“ sagte sie auf
einmal.

[s 9]
„Gern!“ Hans Jung wurde ganz rot.

„Das Rotwerden haben Sie noch immer,“ sagte
sie und lachte.

„Das ist eine dumme Angewohnheit,“ sagte
Hans Jung.

„Das ist keine Angewohnheit,“ sagte die Frau,
und er lachte mit ihr und die Schwere, die sich
auf dem See über sie gelagert hatte, war auf einmal
gewichen.

<s>Hans Jung log, was ihm gerade in den Mund
kam: Es war wirklich ein sonderbarer Zufall, daß
sie sich in diesem regnerischen Seewinkel wiedertrafen.
Ganz zufällig war er darauf gekommen, gerade hier
seine Reise zu beenden. Ein Bekannter hatte es
ihm empfohlen, ein Maler, auf dessen Urteil er sich
verlassen konnte.

Was er alles hier tun wolle? – Nichts als ruhen,
vielleicht ein paar Kompositionen, die schon skizziert
waren, ausarbeiten, für jeden Fall ein paar Wochen
lang still und ohne jeden Aufwand leben.

An dem Feldweg, der zur Blankschen Villa
hinüberführte, verabschiedete er sich. Er mußte noch
ein Zimmer suchen, da er nicht ins Hotel wollte.

„Heute abend müssen Sie viel erzählen,“ sagte
sie, als sie ging.

Er sah ihr, hinter einem Baum verborgen, nach
[s 10]
bis sie an dem Zaun ihres Gartens entlang kam.
Sie schritt wie verloren zwischen den abgeblühten
feuchten Wiesen.

Er wandte sich und ging ins Dorf zurück. An
Zimmern war keine Not, an jedem Haus hingen
Zettel. Er ging in eines der letzten Bauernhäuser
am See, die ganz adelig hinter den neuen Stadt­
häusern lagen. Alle Türen waren unverschlossen.
Niemand antwortete, als er in dem halbdunklen Flur
stand und an die erste Zimmertür klopfte. Eine Uhr
ging drinnen schwer hin und her. Er klopfte noch
einmal und ging hinein. In einem Lehnstuhl neben
dem Fenster saß eine alte Frau und sah erst auf,
als er mitten im Zimmer war.

„Ich höre nicht gut,“ sagte sie und lächelte Hans
Jung an und er sah, daß sie ein dickes Buch auf­
geschlagen auf dem Schoß liegen hatte und tief·braune
Augen hatte mitten zwischen vielen Runzeln. „Marie!“[|,]
rief sie, daß es nur so durch das Haus gellte.

„Bravo,“ sagte Hans Jung und lachte.

„Was ist?“[|,] fragte die Frau.

„Schön ist es hier!“[|,] schrie ihr Hans Jung zu und
hörte nicht auf zu lächeln.

Ein Mädchen erschien in der Tür, ein fein­
gliedriges, schlankes Ding.

„Ich möchte das Zimmer sehen,“ sagte Hans
[s 11]
Jung, und er sah, daß sie die gleichen Augen hatte
wie die alte Frau.

„Kommen Sie,“ sagte das Mädchen, „das
Zimmer,“ schrie sie der Mutter ins Ohr, die sie
fragend anschaute.

Es war ein sehr großes, helles Zimmer im ersten
Stock, mit nur wenigen Möbeln; ein großer Holztisch
mit geschnitzten Füßen, ein hohes Bett, eine alte
Kommode mit ein paar Blumenstöcken. Die Fenster
gingen auf den See hinaus. In ihrem Rahmen war
Nebel und das Wasser des Sees und die Schatten
der Dämmerung. Auf dem Tisch lagen ein paar
Anmeldezettel. „Ich bleibe bei Ihnen,“ sagte Hans
Jung und begann sie auszufüllen.

„Wie heiße ich?“[|,] fragte er das Mädchen, das
hinter seinem Stuhl stand. Sie lachte.

„Sie heißen Marie,“ sagte er und lächelte, während
er schrieb.

Sie sah ihm verwundert über die Schulter –
unter „Beruf“ machte er nur einen Strich.

„Wie alt?“[|,] fragte Hans Jung.

Sie lachte.

„Sie sind achtzehn Jahre,“ sagte er, „und wie
alt bin ich?“

„Ich bin zwanzig,“ sagte sie ein wenig langsam,
„und Sie sind dreißig.“

[s 12]
„Fünfundzwanzig,“ schrieb Hans Jung, ohne zu
antworten.

Marie las den Zettel, als er fertig war.

Das „Wohin“ hatte er vergessen. „Wohin geht
Ihre Abreise?“[|,] las sie vor.

„Wohin?“[|,] sagte Hans Jung und sah sie an und
hörte auf einmal auf zu lächeln. „Ich weiß es nicht.“

Auch das Mädchen lächelte nicht mehr.

Hans Jung öffnete ein Fenster, als sie gegangen
war und sah auf den See hinaus. Es waren nur
wenige Schritte vom Haus bis ans Wasser. Ein
kleiner Rasen war dazwischen, auf dem eine einfache
Holzbank zwischen ein paar Büschen stand. Es hatte
angefangen stark zu regnen und ein eintöniges Rauschen
stieg vom See auf. Die Wolken sanken immer tiefer
und lösten sich. Das Bauernhaus am andern Ufer
war kaum mehr zu sehen. Und die Nebel und die
Dämmerung über dem See flossen ineinander über
und zergingen.

Es war dunkel geworden, als er das Fenster
schloß und ging.

Marie wartete im Flur und gab ihm einen großen
Schirm und den Hausschlüssel. Er bedankte sich und
reichte ihr die Hand, ehe er sich wandte.

Sie ging ins Zimmer zurück. „Du sollst nicht
mehr lesen!“[|,] rief sie zur Mutter hinüber. – „Kind?“[|,]
[s 13]
fragte die Frau und ließ die Bibel in den Schoß
sinken. – „Die Augen,“ schrie ihr Marie ins Ohr.
Die Alte lächelte.

Marie zündete die Lampe an und schaute in
ihren gelben Schein. Die Mutter sah in den Regen
hinaus.

Später kam der Bruder vom Bahnhof zurück
und brachte den großen eleganten Koffer des Herrn
und einen lederbeschlagenen Kasten, der wie eine
Baßgeige geformt war. Nachdem sie zusammen ge­
betet und gegessen hatten, ging Marie hinauf und
ordnete das Zimmer des Herrn[ . . . .
|.
<aa>]
Der Regen trommelte nur so auf dem breiten
Schirm, als Hans Jung durch die Felder zur Blank­
schen Villa ging. Es war ganz dunkel geworden.

Ein Bauernmädchen mit dickem, rotem Gesicht
öffnete ihm und führte ihn durch den weiten Gang
in das Speisezimmer. „Frau Doktor kommt sofort,“
sagte sie und schloß die Tür nicht gerade leise
hinter sich.

Hans Jung blieb mitten in dem stillen Zimmer
stehen.

Es war ein kleiner Raum. In der Mitte stand
ein heller Tisch mit zwei Gedecken und einer niedrigen
Petroleumlampe und einer schlanken Kopenhagener
Vase voll Rosen. Sonst war nichts im Zimmer als
[s 14]
ein kleiner Rauchtisch draußen im Schatten und eine
Ottomane und ganz hinten ein schmaler, hoher Bücher­
schrank. Und ein paar Radierungen an den holz­
getäfelten Wänden.

Und irgendwo eine Uhr, die gerade tief und
lang zu schlagen begann.

Hans Jung ging zu der Ottomane hinüber und
setzte sich ins Halbdunkel. Er starrte vor sich hin
und sah nach dem Tisch mit den beiden Gedecken
und wartete. Er hörte auf einmal die Uhr leise
hin und her gehen.

Frau Marie kam herein und gab ihm die Hand.
Sie hatte erst noch ihren Buben zu Bett bringen müssen.
Den wollte sie ihm ein anderes Mal zeigen.

„Das sind schöne Rosen,“ sagte Hans Jung, als
sie sich an den kleinen Tisch gesetzt hatten. Es war
nur gesagt, um nicht ins Schweigen zu kommen.

Frau Marie erzählte von den vielen Rosen in
<l stadtgarten>ihrem Garten in der Stadt, die heuer zumeist am
Stock verblühen mußten. Ihr Mann schnitt sie nicht
ab und ließ sie lieber im Garten verwelken.

Das Mädchen mit dem roten Gesicht brachte
das Essen und sah dabei Hans Jung an, wie man
einen Eskimo betrachtet.

Er begann von dem Zimmer zu erzählen, das
er gemietet hatte, von dem Blick auf den See und
[s 15]
von der schwerhörigen, alten Frau mit den jungen,
braunen Augen.

„Haben Sie Ihr Cello dabei?“[|,] fragte Frau Marie.

Ja, er hatte es dabei, und sie schwiegen wieder,
und Hans Jung führte die Bissen langsam und
mechanisch zum Mund und sah nicht und schmeckte
nicht, was er aß.

Aber als abgedeckt war, konnte er auf einmal
wieder sprechen, ohne seine Stimme wie etwas
Fremdes von den Wänden zurückklingen zu hören.

Ja, das Cello hatte er dabei und ein paar Bücher
und Notenpapier. Er lachte. Sonst hatte er nichts,
gar nichts. Alles war noch eingepackt wie vor seiner
Abreise. Und er hatte keine Verpflichtungen, keinen
Diener, keinen Hund und kein Pferd – – wie
wenig Ballast! Er lachte wieder. Und keinen einzigen
Brief mehr zu lesen und keinen mehr zu schreiben.

Frau Marie sah auf. „Wollen wir den Tee im
Erkerzimmer trinken?“ – Sie gingen in ein großes
helles Zimmer, in dem ein Flügel war und ein
kleiner erhöhter Erker mit einem runden Tisch. Der
Regen schlug an die Fenster, als sie sich setzten.

„Keinen Brief mehr,“ sagte Frau Marie und
lächelte, „schreiben Sie nicht mehr Ihrem Freund?“

„Freund?“ Hans Jung sah sie an, „ach, Sie
meinen Mendel. Sie haben ein gutes Gedächtnis!“

[s 16]
Er lächelte. „Ich habe ihn schon ganz vergessen.
Wir hielten uns wirklich drei Jahre lang für Freunde.
Das ist lang genug in unserem Alter.“

Er brach ab, obgleich er sah, daß sie saß, als
warte sie, bis er weiter spräche. Jetzt erst sah er,
daß sie den einzigen Schmuck, den er ihr geschenkt
hatte, an einem dünnen Goldkettchen auf der Brust
hängen trug; eine kleine weiße Gemme aus Venedig.

Wie gerne hätte er still gesessen und geschwie­
gen, die Sätze kamen ihm so trocken aus dem Halse.
Aber er sah, daß sie wartete und er sagte:

„Wenn er stolz genug gewesen wäre, um zu
wissen, daß unsere Freundschaft nur eine Gastfreund­
schaft sein konnte[,|] – wir hätten uns eines Abends
irgendwo, an einem Busch oder an einem Bach die
Hand gegeben und gesagt: Leb' wohl! – Aber dazu
war er zu säuberlich. Es mußte bei ihm alles richtig
gebucht und zu Ende geführt werden. So wurde der
Abschied sehr dramatisch. Lassen wir's, nicht?“

„Wollen Sie es mir nicht erzählen?“[|,] sagte Frau
Marie, ohne aufzuschauen. Sie hatte den Kopf ein
wenig tiefer gesenkt, es war als horche sie auf etwas
sehr Leises.

„Es ist nicht viel zu erzählen,“ sagte Hans Jung.
„Und es ist wirklich das erstemal, daß ich mich recht
daran erinnere. Er hatte selbst gemerkt, daß unsere
[s 17]
Straßen auseinandergingen. Aber als ich sagte: »Grüß
Gott und einen schönen Gruß an Deine Braut,«[ –|]
suchte er nach etwas Besserem, vielleicht, weil ich ein
wenig gelächelt hatte –“

Frau Marie hatte den Kopf gehoben und sah
ihn an. Aber sie lächelte nicht mit ihm.

„»Schuft«, sagte er, ehe er ging. Es war ein
guter Abgang. Aber ich sang auf dem Heimweg
und es fiel mir eine Melodie ein, die mir damals
gut gefiel.“

Er schwieg und sah auf die Gemme auf ihrer
Brust. Wie er sich noch genau an den braunen Kerl
erinnerte, von dem er sie an der Riva degli Scia­
voni gekauft hatte. Nichts war in den Stein ge­
schnitten als eine venetianische Gondel, ohne In­
sassen und ohne Gondolier – nur der schlanke Bogen
der Gondel und ein paar Wellenlinien, das Wasser.

<s 0.16>„[|\kern -0.03em]Kommen[ | ]Sie[ | ]nie[ | ]mehr[ | ]mit[ | ]ihm[ | ]zusammen[|\kern -0.1em]?[|\kern -0.1em]“[|\kern -0.1em][|,][
| ]fragte[ | ]Frau[ | ]Marie[|\kern -0.05em].<s 0>

„[|\kern -0.02em]Ich weiß es nicht,“ sagte Hans Jung, „es ist
mir gleichgültig.“ Und er besann sich, daß er es fast
ärgerlich gesagt hatte und fuhr leiser fort: „Darf ich
das nächstemal das Cello mitbringen? Und Ihnen ein
paar Sachen aus Asien zeigen, die ich mit hierher
gebracht habe? Und wollen Sie nicht ein wenig singen?
Oder wollen Sie nicht, heute?“

[s 18]
Und Frau Marie saß am Flügel, und er setzte
sich auf einen Stuhl hinter sie, tief ins Dunkel und
horchte auf ihre ruhige, weiche Stimme, ohne auf die
Worte zu achten.

Und er ließ den Kopf tief sinken.

Sie sang Schubert=Lieder.

„Zieh weiter, Gesell, zieh weiter – –

Es rauschen Mühlenräder in jedem tiefen Tal –“

Und dann die schönsten Lieder aus den letzten
Bänden.

Hans Jung hob den Kopf und sah auf den
Rücken der Frau. Sie hatte die Schultern leicht ge­
beugt und das Haar war glatt aus dem Nacken
emporgekämmt. Als die letzten Töne verklungen
waren, horchte sie noch . .

Und dann verabschiedete er sich, und sie gab
ihm die Hand unter der dunklen Haustür, und er
ging wieder unter seinem großen Schirm durch die
nassen Wiesen, und das Rauschen stieg noch in der
gleichen Eintönigkeit vom See auf, als er in seinem
neuen Zimmer lag und in das Dunkel emporstarrte[. . .
|.
<a>]
Er erwachte aus tiefem Schlaf, und der Regen
hatte aufgehört zu rauschen. Es war schon heller Tag.

Er blieb liegen und sah sich im Zimmer um,
ohne den Kopf zu bewegen.

Dort lag sein Koffer und das Cello, und auf dem
[s 19]
Tisch stand ein Krug mit einem dichten Strauß
Wiesenblumen. Das große Mädchen hatte sie ihm
hereingestellt, das sich immer eine blonde Strähne
aus der Stirn streichen mußte, wenn es lachte. Marie
hieß sie.

<l marie_ppp>Marie . . .

Aus dem kleinen Garten hinter dem Haus
drang auf einmal das langsame Hin= und Her­
schnarren einer Säge herauf.

„Hör' auf, Thomas. Der Herr schläft noch!“
Es war die Stimme Maries. Sie rief aus einem
Zimmer dicht neben dem seinen.

„Nein, er schläft nimmer!“[|,] rief Hans Jung,
<h „guten> Mor­gen!“ – „Guten Morgen!“[|,] rief der
Bruder her­auf. – „Guten Morgen!“[|,] rief es aus dem
Zimmer neben ihm. Sie lachte wieder kurz auf, wie
gestern. Vielleicht strich sie sich wieder die Strähne
in den glatten Scheitel zurück.

Hans Jung schloß die Augen wieder, ohne ein­
zuschlafen, und die Säge ging weiter hin und her.
Dann hörte sie auf einmal auf. Nach einiger Zeit
klang leis aus der Stube unter ihm der Gesang einer
tiefen weiblichen Stimme durch den Holzboden herauf.

Es war Maries Stimme.

Manchmal war sie eine Weile still. Die Tür
ging auf und zu, und es wurde ein wenig gesprochen.
[s 20]
Dann sang sie weiter in der gleichen eintönigen Me­
lodie, die wenig in die Höhe und wenig in die
Tiefe ging.

Hans Jung richtete sich auf, um die Worte ihres
Liedes verstehen zu können.

Sie sang:

<v begin>
Die Mutter sprach zu Barbara:\\*
Es ist Musik vorm Haus.\\
Sind viele schlanke Burschen da,\\
Geh auch zum Tanzen, Barbara,\\
Such' einen Bräutigam aus!\\
Deine Brüstlein sind schon rund,\\*
Barbara!

Da ging die stolze Barbara\\*
Zur Tanzmusik hinaus.\\
So viele Freier waren da,\\
Doch keinen wählte Barbara\\
Sich zum Herzliebsten aus.\\
Hast einen Rosenmund,\\*
Barbara!

Die Zeit verging, und Barbara\\*
Kam nie als Braut nach Haus,\\
Sie saß als alte Jungfer da\\
Und sah durchs Fenster. Barbara,\\
Es schneit und regnet draus!\\
Welk sind die Brüstlein rund,\\*
Barbara!
<v end>

[s 21]
Hans Jung horchte noch, als das Lied schon
lange zu Ende war.

Aber es wurde nimmer gesungen.

Er sprang aus dem Bett und stand nackt und
straff am Fenster. Er sah auf den See, auf dem noch
immer schwer und tief die Wolken hingen. Als Marie
mit dem Frühstück kam, war schon der Koffer aus­
gepackt und sein Inhalt in dem Schrank und den
tiefen Fächern der Kommode geordnet. Oben auf
der Kommode lag eine chinesische Seidendecke und
eine kleine Reihe Bücher und die dicke Mappe mit
dem Notenpapier.

Sie breitete das Geschirr vor ihm aus, und als
sie ihn fragte, wie er geschlafen habe, stieg ihr auf
einmal das Blut zu Gesicht. Er wollte sie ein wenig
zurückhalten und sagte, als sie schon bei der Tür
war: „Ein schlechtes Ge­wissen ist ein gutes Ruhe­
kissen.“ Marie lachte, heute hielt die Strähne. –
„Haben Sie ein schlechtes Gewissen?“[|,] fragte sie.
Sie sprach ein wenig gezwungen hochdeutsch, und
das stand ihr gut. – „Immer,“ sagte er und lachte.
Aber sie hatte schon die Türklinke in der Hand. –
„Und ich danke schön für die Blumen,“ sagte er.
„Die sind fein gebunden.“ Sie knickste ein wenig
und ging hinaus. Hans Jung sah ihr lächelnd nach,
als die Tür schon lang wieder geschlossen war.

[s 22]
Später übte er auf dem Cello und sah auf den
See hinaus, auf dem die Wolken lichter wurden.
Und vor dem Mittagessen ging er in die Stube
hinunter, in der Maries Mutter saß und ihn nicht
anklopfen und nicht eintreten hörte und ihn lächelnd
begrüßte, als er auf einmal vor ihr stand. Er be­
zahlte das Zimmer für einen Monat voraus und ließ
sich auf seinen Schein, der fast das Doppelte machte,
nichts zurückgeben. Die alte Frau war ein wenig
beschämt, aber dann mußte sie lachen über die Lustig­
keiten, die er ihr ins Ohr schrie.

Hans Jung ging zum Mittagessen in das große
Hotel. Die ganze Veranda war voll Fremder. Die
meisten waren in Gebirgstracht, die Damen hatten
nackte Arme und enge Mieder, aber oft sah man
Korsettspangen dort, wo bei Marie die kleinen
Knospen in das Tuch stachen. Hans Jung fiel durch
die stille und eherne Eleganz auf, die er an sich
pflegte. Er fand Platz an einem Tisch, an dem ein
reiches Ehepaar mit ihrer Tochter saßen. Wie
Kuppler neben ihrer Ware saßen sie neben ihrer
Miedertochter, der Vater mit Lederhosen und schwarz­
gerändertem Zwicker, die Mutter geschnürt und ge­
pudert. Hans Jung begann während des Essens mit
dem Töchterchen, das einen orientalisch offenen Blick
hatte, zu kokettieren, und er senkte hastig und beschämt
[s 23]
die Augen, wenn sie sich angesehen hatten und bückte
sich schnell nach der Serviette, die ihr versehentlich
neben ihm entglitt, worauf sie dankend nickte und
die Mutter wohlwollend lächelte, und als er ging,
grüßte er tief und ernst, daß alle drei strahlten.

Zum Kaffee sollte er mit dem Cello zu Frau
Marie kommen. Es war noch zu früh, er ging ein
wenig dem Seeufer entlang und sah auf das Wasser,
auf dem wechselnd wieder die Sonne lag.

Die Wolken gingen wieder hoch.

Das Mädchen mit dem roten Gesicht führte
ihn in das Zimmer, in dem der Flügel stand. Die
gnädige Frau wäre bei dem Herrn Baron drüben
und käme sofort. Hans Jung wußte nicht, wer der
Herr Baron war, er lehnte sein Cello an den Flügel
und schickte sich an, zu warten.

Auf einem kleinen Ebenholztisch hinter dem
Flügel stand eine Plastik, eine kleine Bronzegestalt,
die er gestern nicht bemerkt hatte. Eine nackte
Tanzende, schlank, fast unnatürlich schlank und mitten
in einem Wirbel erstarrt. Wie aufreizend die Linie
war, den Hals und den gebogenen Rücken herab.
Der Kopf war tief gebeugt, und ein Verhalten war
in den Gesichtszügen, als schäme sie sich. Und die
Brüste waren kindlich klein.

Er wandte den Blick und sah auf das wirre
[s 24]
Durcheinander der Blätter und der Zweige des
Baumes, der sich bis an die Fenster des Erkers drängte.
Die Vögel sangen darin, er horchte zu und wartete.

Es polterte an die Türe, die Klinke schnappte
ein paarmal ein, und ein kleiner Bub schob sich
durch den Spalt. Als er den Fremden sah, wäre
er am liebsten wieder davongelaufen. Aber es
war schon zu spät. Hans Jung schloß ihm die Tür
und hob ihn auf den Arm.

„Komm,“ sagte er, „ich will dir etwas erzählen.“

Er trug ihn zum Sofa hinüber und setzte
ihn auf seine Knie. Der Bub sah ihn immerzu
verwundert an – er sah ihn an mit Augen, denen
nur der Schein von zwanzig langen Jahren fehlte,
dann wären es Frau Maries Augen gewesen.

„Wie heißt du denn?“[|,] fragte Hans Jung. Der
Bub wandte den Blick nicht von ihm und gab ihm
keine Antwort.

Und Hans Jung dachte daran, daß er Frau
Marie gesehen hatte, als sie an dem Buben trug. . .
Es war in dem Sommer gewesen, in dem er sie
kennen gelernt hatte und an dessen Ende er ab­
gereist war. In den Tagen, in denen der Sommer
starb, war er gegangen.

„Nun gehe ich bald,“ sagte er, als er mit ihr
auf der großen Blankschen Veranda gesessen war
[s 25]
kurz vor der Abreise. Sie warteten zusammen auf
ihren Mann, der noch in seiner Klinik beschäftigt war.

Er hatte ihr Rosen gebracht, und sie hatte
ausgesehen in ihrer Schwangerschaft wie eine Heilige.

„Wollen Sie wirklich jahrelang fortbleiben?“[|,]
hatte sie gefragt. Aber sie hatte nicht gewußt, was
sie ihm mit dem Klang ihrer Worte mit auf den
[den|] Weg gab – „dann werden Sie uns bald ver­
gessen haben“.

Hans Jung antwortete nicht.

„Wir kennen uns ja so kurz,“ sagte sie.

<l entblätterte>Sie nahm eine von den Rosen und entblätterte
sie, ohne es zu wissen.

Sie schwiegen, bis der Doktor kam.

Er sprach von der Klinik und von seiner großen
Arbeit, und als Hans Jung ging, sagte er: „Na, bis
Sie uns dann wieder besuchen, wird es ein wenig
anders bei uns aussehen,“ – und er konnte sich
nicht versagen, dabei einen Blick auf den blühenden
Leib der Frau zu werfen, daß sich Hans Jungs Fäuste
in den Manteltaschen ballten. . .[
|<aa>]
„Fritz,“ sagte der Bub auf seinem Knie auf
einmal. Es klang so zaghaft.

„Fritz heißt du?“[|,] und er wippte ihn auf seinem
Knie auf und ab.

„Soll ich dir etwas erzählen?“

[s 26]
Der Bub nickte.

<l märchen_anfang>„Ein Märchen?“ – Aber er wußte keine Mär­
chen mehr. Er mußte erst eines für ihren Buben
ersinnen. Er mußte erzählen, was ihm gerade in
den Mund kam, und es kam ihm dieses Märchen
in den Mund:

„Es war einmal eine Frau, die war wie ein
Baum; wie ein Baum, gleich jenem dort vor dem
Fenster, aber schlanker und tausendmal schöner und
mitten auf einem weiten Feld, das kahl war. Der
Baum war voll seltener blauer Blüten und sein Stamm
war glatt und hell. Die häßlichen Vögel flogen daran
vorbei und sahen ihn nicht. Die schöneren Vögel
flogen daran vorbei und sahen sehnsüchtig nach seinen
Blüten und durften nicht darinnen ruhen. Denn nur die
schönsten Vögel durften in seiner Krone bleiben. Und
sie sangen so lange zwischen den Blütenzweigen, bis
sie tot und schwer zu Boden fielen. Der schlanke
Baum aber zitterte nur ein wenig, wenn sie starben
und blühte ruhig weiter, immerzu.

Aber einmal kam doch ein häßlicher Vogel in
den Baum und schrie in den Blütenästen wie eine
Krähe, und von dem Tage ab flogen auch die schönsten
silbernen Vögel traurig vorbei und starben anderswo.
Ein einziger fremder Vogel kam aber eines Abends
von weitem her und setzte sich dem häßlichen gegen­
[s 27]
über. Er war vollkommen schön und saß auf einer
langen, blauen Dolde und sang, daß sich sein eigenes
Gefieder sträubte vor Lust. Da schrie der häßliche
Vogel immer leiser, bis er herunterfiel, tot, und der
Wind ihn über die dunklen Felder schleifte. Aber
der andere sang weiter, allein, und der Baum trieb
unter ihm neue Blütendolden, immer mehr, bis er
da stand, daß einem die Brust schwer wurde, wenn
man in die Nähe kam.

Und eines [Morgen|Morgens] fiel auch der schöne Vogel
tot aus den Blüten. Der Baum stand leer.

Da verblühte er. Die Dolden wurden braun
und gelb und dörrten ab und fielen. Die Äste
wurden dürr. Und der Baum stand alt in dem
kahlen Feld[.|] –“

Es war ein richtiges Märchen geworden, denn
der Bub verstand kein Wort von dem, was der
Fremde sagte und bekam nur einen verträumten Blick.<l märchen_ende>

„Hopsasa, hopsasa,“ machte Hans Jung und
schüttelte ihm das Träumen aus den Augen.

„Weißt Du, woher ich jetzt gekommen bin?“[|,]
fragte er.

Der Bub sah ihn unverwandt an.

<l klischee_china>„Jetzt war ich bei den Chinesen. – Das ist
über einem großen, großen Meer. Und dort haben
alle Leute lange Zöpfe.

[s 28]
Geschlitzte Augen und gelbe Gesichter und lange,
lange Zöp­fe[.|] –“

Die Tür ging auf, Frau Marie kam und mit
ihr ein älterer Herr, der sehr elegant gekleidet war,
glatt rasiert und schmal und lang.

Er gab Hans Jung die Hand, lose und ohne
Druck, und sein Gesicht, das unter der Tür noch
glatt und jugendlich erschienen war, zerfiel in der
Nähe in viele Furchen und Runzeln. „Baron Man­
nen“ oder etwas Ähnliches hatte Hans Jung verstanden.

Sie setzten sich zusammen zum Kaffee in den
Erker. Der Bub lehnte sich an den Stuhl seiner
Mutter. Noch immer sah er Hans Jung an. Der
Fremde schwieg und sah sehr stolz aus. Hans Jung
betrachtete ihn, wenn er vorsichtig an seiner Tasse
sog. Dann sah man nur die hohe Stirn und das
dichte, graue Haar.

„Ich hab' ihm von den Chinesen erzählt,“ sagte
Hans Jung, „drum schaut er noch so erstaunt.“ Er
lachte und sagte ohne jede Rücksicht auf den alten
Mann, dessen selbstbewußtes Schweigen ihm ärgerlich
war: „Er hat die gleichen Augen wie Sie – –
oder vielmehr, er bekommt sie vielleicht später einmal.“

Er wußte, daß er es bei ihr allein nicht gesagt
hätte.

„Ich habe schon gehört, daß Sie von so großen
[s 29]
Reisen kommen,“ sagte der Baron. Er sprach ein
klein wenig durch die Nase.

Was war zu antworten? – „Ja,“ sagte Hans
Jung und sah in die stahlblauen, alten Augen, die
ihm ruhig auswichen.

„Das ist das Beste, was junge Leute tun können,“
sagte der Baron.

„Das kommt wohl auf den einzelnen an,“ sagte
Hans Jung.

Der alte Mann lächelte ein wenig. „Es tut
einem jeden gut,“ sagte er, „Sie können es mir
glauben.“

Hans Jung sagte lächelnd und sah dabei auf Frau
Maries Hand, die schlank auf dem Tische lag:
„Glauben Sie nicht, daß vielen das Reisen auch nichts
mehr nützen kann? Ich finde, es kommt immer nur
auf den einzelnen an.“

Der Alte hatte nicht aufgehört zu lächeln.
„Wenn es überhaupt in Ihrem Alter viele »einzelne«
gibt,“ sagte er.

Hans Jung zog seine russische Holzdose aus
der Tasche und sagte statt einer Antwort liebens­
würdig: „Darf ich Ihnen eine Zigarette anbieten?“

Der Baron dankte.

Frau Marie nahm eine Zigarette an und lächelte
dabei leis und kaum merkbar. Sie blickte langsam
[s 30]
auf, Hans Jung beobachtete sie. Jetzt schaute sie ihn
an, und er sah, daß das Lächeln für ihn gewesen war.

Sie rauchte wie alle Frauen, die selten rauchen
und blies den Zigarettenrauch viel zu stark aus.
Hans Jung sagte: „Sie bearbeiten die Zigarette wie
eine kleine Trompete,“ und alle lachten, und sie
sprachen vom Rauchen bei Frauen, das manchen gut
anstehe und manchen nicht – „wie eine Lorgnette
oder ein kleiner Flaum auf der Lippe,“ sagte der
alte Mann, wie um das Thema abzubrechen.

„Geh zu Anna, Fritz,“ sagte sie, und der Bub sah
noch einmal groß zu Hans Jung hin, ehe er sich trollte.

„Es war ihm noch nicht faßbar,“ sagte sie, „daß
Sie von den Chinesen kommen.“

„Es muß wunderbar sein,“ sagte der Baron
langsam, „noch wie ein Kind die Möglichkeit zu
unbegrenztem Träumen zu haben.“

„Das ist wahr,“ sagte Hans Jung.

„Unser vielgepriesenes Wissen tötet uns diese
Möglichkeit ganz ab,“ sagte der Baron, und Hans
Jung merkte, daß er sich nur an Frau Marie wandte.

„Man kann sich ja von seinem Wissen trennen,“
sagte Hans Jung, „so oft man nur will.“

„So oft man will, kaum,“ erwiderte der alte
Mann und lächelte wieder. „Aber manche Leute
manchmal. Das gibt dann die Dichter.“

[s 31]
„Ja,“ sagte Hans Jung in höf·lichem Tone und
sah dabei auf den venetianischen Schmuck, der auch
heute auf ihrer Brust hing, „oder es gibt jene Frauen
und Männer, die immer, und wenn sie hundert Jahre
alt werden, weich wie Kinder bleiben.“

Der alte Mann lächelte noch immer spöttisch,
aber Hans Jung sah auf einmal, daß das Lächeln
bei ihm nur eine erstarrte Maske war.

Frau Marie bat ihn, etwas auf dem Cello zu
spielen.

„Wissen Sie noch, wie Sie das letztemal bei
uns gespielt haben,“ sagte sie, „wie lang es her ist[“ . . .| . . .“][
|<a>]
Ja, es war lange her. Es war an dem Abend
gewesen, bevor er sich verabschiedet hatte. Damals
war er gut im Spielen gewesen. Er hatte eine Bach­
sonate gespielt und dann eine seiner ersten eigenen
Kompositionen, eine kleine Elegie. Und er hatte
gesehen, wie Herr Doktor Blank nur mit Mühe die
Augen offen hielt; er hörte gute Musik nur deshalb
gern, weil ein gebildeter Mann mit künstlerischen
Interessen und einer musikalischen Frau das tun
muß. . . . Frau Marie aber war wach gewesen –
wach. Er hatte ihr nicht gesagt, daß die Elegie von
ihm war. Sie war still in ihrem tiefen Stuhl gesessen
und hatte ein langes, blaues Kleid getragen, das
ihre Schwangerschaft nicht verbergen sollte[. . .
|.
<a>]
[s 32]
„Es geht jetzt sehr schlecht mit dem Spielen,“ sagte
Hans Jung. „Wenn man so lange nicht mehr ge­
spielt hat.“

„Haben Sie auf Ihrer Reise nie gespielt?“[|,] fragte
der Baron.

„Nein, ich hatte mein Instrument nicht dabei,“
er lächelte, „einmal nur habe ich ein wenig gespielt
in einer kleinen Hotelkapelle. Das war sehr lustig.
Ich gab dem Cellisten ein Trinkgeld und das Ver­
sprechen, ihn nicht blamieren zu wollen, und er über­
ließ mir gern für den ganzen Abend seinen Platz.
Es waren fünf Ungarn.“

„Wo war das?“[|,] fragte Frau Marie, „in Ungarn?“

<s>„Nein, in Kairo. In einem der großen Hotels.
Und die Leute der Hotelkapelle waren angenehmer
als die reichen Gäste. Aber in einer Tanzpause
spielte ich den Gästen ein Solo vor. Ich entschuldigte
mich im voraus bei den Ungarn und spielte mit
Klavierbegleitung: [„|»]O du mein holder Abendstern[“|«]
und tremolierte durch den prunkhaften Saal, daß das
Eis, das die Damen in kleinen Kristallschalen vor
sich hatten, nur so dahinschmolz.“

Sie lachten zusammen.

Aber dann wandte Frau Marie ein wenig den
Kopf und sagte vorsichtig: „Oder Sie hätten etwas
Besonderes spielen können, was niemand verstanden
[s 33]
hätte, und es wäre vielleicht so kostbar geworden,
als ob Sie Ihr Cello in die Wüste getragen hätten.“

Sie brach ab.

„Ich will es gern ein wenig versuchen,“ sagte
Hans Jung, „wenn es Ihnen recht ist.“

Er mußte ohne Noten spielen, und er versuchte
eine Allemanda von Bach, die er gut kannte.

Er hatte sich neben den Flügel gesetzt mit der
Seite gegen den Erker, und er wußte, daß man von
dort aus sein Profil sah, wenn er den Kopf über den
Hals seines Cello neigte. Aber dann versank er in
den kostbaren Ton seines alten, verwittert braunen
Instruments.

Frau Marie hatte ihren Sessel vom Tisch ge­
schoben, daß sie neben den Baron zu sitzen kam.
Sie hatte den Kopf geneigt, wie es ihre Gewohnheit
war, wenn sie horchte. Und ihre Arme lagen wie
erschlafft auf den kleinen Seitenlehnen des Sessels.

Als er den letzten Ton zu Ende gezogen hatte
und schon wieder stimmte, hob der Baron auf einmal
den Kopf und sagte: „Ich danke Ihnen“[|,] – mit einer
Stimme, die nimmer zu erkennen war, jung und weich.
Hans Jung mußte noch das Gesicht des alten Mannes
ansehen, als er es schon wieder horchend senkte.
Wie seine Züge ruhig und angenehm waren, ohne
jenes starre Lächeln.

[s 34]
Er spielte eine kleine Sarabande von Haydn
und dann, er wußte es selbst kaum, wie er darauf
verfiel, eine kleine Romanze, eine seiner letzten eigenen
Kompositionen, die nur skizziert war. Fertig war
nur eine kleine führende Melodie; alles andere mußte
er improvisieren.

Aber die Melodie kam immer wieder. Ein selt­
samer, kurzer Lauf, mit tiefen Doppelgriffen beginnend,
dann hoch hinauf getragen und wieder herab bis zur
Mitte, wo sie zur Ruhe kam. Dort wo sein Cello
am vollsten sang, ließ er sie ruhen.

Er wandte lächelnd den Kopf, ob sie merkten,
daß er den Schluß nicht finden konnte. Frau Marie
hatte den Kopf gesenkt, wie vorhin . . . und der
alte Mann hatte sich gewandt und sah sie immerzu
an und strich langsam über ihre Hand, die auf der
Sessellehne lag . . . langsam hin und her, bis zu
dem hellen Saum ihres Sommerkleides und zurück
bis zu den Fingerspitzen . . . hin und her mit seiner
mageren, alten Hand, an der ein großer, matter Perl­
ring war. Und sie ließ es still geschehen.

Hans Jung wandte den Kopf zu seinem Cello
zurück und spielte weiter. Er kam wieder in die
Melodie, die mit den tiefen Doppeltönen begann und
hoch hinauf und wieder herab zur Mitte ging. Aber
er ließ sie nicht ruhen, er spielte auf dem letzten Ton
[s 35]
einen langen Triller und einen Doppelschlag und süß,
tremolierend, unverschämt: „O du mein holder Abend­
stern!“ Und dann brach er mitten in dem Beten ab.

„Fein übergeleitet,“ rief der Alte und sprach
wieder ein wenig durch die Nase, „ich glaube wohl,
daß das den Engländern gefallen hat.“

Hans Jung stellte lächelnd sein Instrument zur
Seite und kam zum Erker zurück.

„Aber warum haben Sie die schöne Melodie
damit totgeschlagen?“[|,] fragte Frau Marie.

Hans Jung sah sie fest an. „Es geht vielleicht
ein anderesmal etwas besser,“ sagte er.

Der Baron mußte sich verabschieden. „Wir
werden uns ja jetzt öfter sehen,“ sagte er zu Hans
Jung und sprach wieder in jenem liebenswürdigen
Tonfall. Frau Marie begleitete ihn nicht; er ging
allein, wie ein Gast, der oft im Hause ist, nachdem
er ihr die Hand geküßt hatte.

Sie schwiegen ein wenig, als der schmale, lange
Rücken durch die Tür verschwunden war.

„Nun ist es erst richtig Sommer geworden,“
sagte Frau Marie. Sie nahm eine von den Rosen,
die auf dem Tisch standen und entblätterte sie langsam,
Blatt für Blatt. Die Sonne kam durch die Fenster,
draußen im Garten sang eine Amsel mitten unter
dem schrillen Spatzenpfeifen.

[s 36]
Frau Marie lächelte. „Wollen Sie nicht jetzt
noch ein wenig spielen?“

Er sah auf und lächelte mit ihr. „Es ging
wirklich schlecht,“ sagte er, „ein anderesmal geht es
besser.“

Als er ging, begleitete sie ihn ein Stück ohne
Hut durch die Wiesen.

„Ich kenne Mannen erst seit diesem Sommer,“
sagte sie auf einmal, „er ist einer der seltsamsten
Menschen, denen ich begegnet bin.“

Hans Jung ging schweigend neben ihr.

„Ich glaube, er hat ein großes Leben hinter sich,“
sagte sie.

Er sah ihr wieder nach, hinter dem alten Baum
verborgen, wie sie langsam und versonnen durch die
Wiesen schritt.

Stufe um Stufe knarrte, als er die Holztreppe
zu seinem Zimmer hinaufging. Es waren ein paar
wilde Rosenzweige zu den Feldblumen auf dem Tisch
hinzugekommen, die Streifen der Abendsonne lagen
auf dem See.

Er nahm ein kleines schwarzes Heft aus der
Mappe auf der Kommode und lehnte sich damit ans
Fensterkreuz. Die Dämmerung schlich sich ins Zimmer,
während er langsam zwischen den Seiten blätterte.
An manchen Stellen war die Schrift verwischt und
[s 37]
kaum zu lesen, und manchmal war die Sonne darauf
gekommen und hatte das Papier vergilbt.

[|\textit{]„ . . . 14.\ April. Heute ist hoher Seegang.
Am Mittag fuhren wir an einer kleinen Insel vorbei,
die auf keiner Karte angegeben ist. Ein Schwarm
Möwen flog lange mit von ihr. Ich war noch nie
auf einem stilleren Schiff. Der vornehme Chinese
kam gerade in meine Nähe, als ich unter dem Geländer
an Deck saß und die Beine über das Wasser hängen
ließ. Die Sonne war im Untergehen. Aber er sprach
mich nicht an, als er mich sah; er entfernte sich stumm
mit einer Scheu vor der heiligen Stille rings, die nur
wenigen Abendländern gekommen wäre . . . Wer
ziehet diese Straß' – Maries Gruß nicht unterlass'.[|}]

[|\textit{]16.\ April. Heute abend sollten wir an Land
kommen. Aber es wird nacht werden bis dorthin.
Wie ruhig die See am Abend liegt. Ich war den
ganzen Tag an Deck, um das Meer zu trinken, das
morgen hinter mir sein wird.[|}]

[|\textit{]17.\ April. Noch immer sind wir mitten im Meer . .
wer ziehet diese Straß' – Maries Gruß nicht unter­
lass'.“[|}]

Hans Jung schloß das Heft und starrte auf das
dämmernde Wasser hinaus.

Es klopfte schüchtern an der Tür; Marie kam
und brachte eine Petroleumlampe für den Abend.

[s 38]
„Wenn Sie lesen wollen,“ sagte sie und ordnete
den Tisch.

„Lesen Sie auch?“[|,] fragte Hans Jung.

Sie nickte nur.

„Was?“

„In einem Buch, das mir ein vorjähriger Sommer­
gast geschenkt hat,“ es klang fast wie eine Ent­
schuldigung: „ein alter Professor.“

„Was ist es?“[|,] fragte er und lächelte, ohne zu
wissen warum.

„Die Irrfahrten des Odysseus,“ sagte sie und
lehnte sich ein wenig an seinen Tisch, „von Homer“.

Hans Jung lächelte und sah unverwandt in die
warmen braunen Augen, die seinem Blick nicht aus­
wichen.

„Es muß ein feiner Mann gewesen sein,“ sagte
er, „der alte Professor.“ Er sah ihr nach und sah
noch die schlanke Falte, die der Rücken ihres Mieders
zwischen den Schulterblättern warf, als sich die Tür
schon lang wieder hinter ihr geschlossen hatte.

Als er durch den dämmerigen Flur schritt, um
zum Abendessen zu gehen, hörte er sie in der Stube
beten; die laute Stimme des Bruders, der vor sich
hinleierte wie in der Schule und die der Mutter, fest
und tief, und weich und schwank dazu Maries Gebet.
„Und vergib uns unsere Schulden, daß auch wir ver­
[s 39]
geben unseren Schuldigern . . .“ Der Dreiklang sang
bis auf die Straße hinaus. Zwischen den Häusern
glänzten lange Stücke des Sees, und die Straße war
still und leer. Jetzt waren sie alle daheim und beteten
das gleiche.

Hans Jung blieb stehen und hob die Arme.
„Vergib mir nicht meine Schulden,“ sagte er lächelnd
zu dem reinen, dämmernden Himmel hinauf. – Es
war nur, um die Brust ein wenig zu recken.

Als er später in den Flur zurückkam, in dem
eine kleine Lampe brannte, klang das Spiel einer
Mundharmonika aus der Stube.

<s 0.22>Sie saßen um den hohen Eichentisch, Marie neben
dem Bruder auf der Bank mit der lustig geschnitzten
Lehne, die Mutter in ihrem tiefen Sessel. Er mußte
sich ein wenig zu ihnen setzen, Marie brachte ihm
einen Stuhl. Die Mutter legte die Hände in den
Schoß und suchte angestrengt ihn zu verstehen und
lächelte dabei.<s 0>

Der Bruder scheute sich ein wenig, vor Hans
Jung weiter zu spielen. Da holte er seine „Baßgeige“
aus seinem Zimmer und spielte ihnen vor. Ein paar
Volkslieder, die sie kannten und ein paar Tänze,
die er einmal in Böhmen gelernt hatte. Der Bruder
sah mit offenem Munde zu, und die alte Frau faltete
die Hände im Schoß ihres Kleides und sah auf die
[s 40]
Finger, die auf dem Griffbrett tanzten. Und Marie
hatte den Kopf geneigt wie Frau Marie, wenn sie
horchte, und strich die Strähne, die heruntergefallen
war, nicht aus der Stirn.

„Jetzt machen wir zusammen Konzert“, sagte
Hans Jung zu dem Bruder, der schwerfällig lachte
und nach der Harmonika griff.

„Konzert!“[|,] schrie Hans Jung zur Mutter hin,
und sie lächelte nur mit dem kleinen, zusammen­
gefalteten Mund.

Der Bruder steckte die Harmonika tief in den
Mund und spielte. „Mäd'le, ruck, ruck, ruck an meine
grüne Seite“ spielte er. Hans Jung vollführte eine
Begleitung in tiefen Terzen und Sexten; und dann
wiederholten sie es, und er spielte in Doppelgriffen,
und das drittemal spielte er es pizzicato und zupfte,
daß es wie von hundert kleinen Trommeln klang.

Das Mädchen saß gebeugt und ohne aufzusehen,
die Arme auf den Tisch gestreckt. Und die Mutter
beobachtete das Gesicht des Fremden, während er
sich über den Hals des Cello neigte, sie betrachtete
es aufmerksam, als suche sie darin. Aber der Bruder
schob das kleine Blechding immer tiefer in den Mund
und lachte unter dem Spielen und wiederholte sein
Lied immerzu, bis Hans Jung mit einem Akkord über
alle Saiten Schluß machte.

[s 41]
„Jetzt muß Fräulein Marie singen,“ sagte er,
während der Bruder auf seinem Schenkel das Wasser
aus der Harmonika klopfte.

Marie wurde dunkelrot. „Ich kann nicht singen,“
sagte sie.

„Das glaube ich nicht,“ sagte Hans Jung lächelnd,
und die Mutter hatte den Kopf zur Seite geneigt
und hob die Augenbrauen hoch, als könne sie da­
durch besser verstehen.

„Fräulein Marie soll singen,“ rief ihr Hans
Jung zu.

Ihre Augen begegneten den Augen Maries, die
ihr auswichen. „Nein,“ sagte sie, „aber Sie können
schön spielen.“

„Haben Sie es denn hören können?“[|,] rief Hans
Jung, „können Sie es denn hören?“ Er hatte immer
noch das Cello zwischen den Knien.

Sie nickte und lächelte, und der Bruder hatte
schon wieder die Harmonika in den Mund geschoben
und versuchte einen Walzer.

„Jetzt wird getanzt,“ sagte er und zwinkerte
Hans Jung zu.

Hans Jung tanzte mit Marie, und der Bruder
wiederholte seinen Walzer immerzu. Die Mutter
folgte allen Bewegungen der beiden, hin und her durch
die große Stube, hin und her mit ihren braunen
[s 42]
Augen, ohne den Kopf zu rühren. Aber der Bruder
begann mit seinen schweren Füßen den Takt mit­
zustampfen, und Hans Jung spürte kaum den Leib
des schlanken Dings, so leicht tanzte sie. Nur ihr
Rückgrat fühlte er in seinen Händen, die er hinter
ihr verschlossen hielt. Sie hielt beim Tanzen den Kopf
zur Seite, still und seltsam.

Und erst auf der Treppe, als er über die
knarrenden Stufen in sein Zimmer zurückging, fiel
ihm ein, daß sie beim Tanzen ausgesehen hatte, als
leide sie.
[|<aa>]
Frau Marie hatte recht – nun war es erst
richtig Sommer geworden.

Der Himmel war rein und blau, und wenn es
Wolken gab, waren es kleine hohe Sommerwolken.
Die Grillen schrien die ganze Nacht, und die Mäd­
chen trugen ihre Brüste wie einen Schmuck. <l traum_anfang>Und
Hans Jung wunderte sich, daß er nicht träumte in
den Nächten – von einem silbergrauen Vogel oder
einer Blume, die nur einmal blühte mitten im Meer,
von einer Frau, die ihre Arme hob und seltsam sang.

Er schlief tief und träumte nicht. Aber die Tage
waren voller Sommerträume . . . Wenn Frau Marie
bei den Rosen im Erker saß, den Kopf gesenkt, die
Hände still im Schoß und horchte. Und wenn er
saß und alte Tänze für sie spielte und versank. . .[
|<aa>]
[s 43]
Es war Abend, sie waren durch die stillen
Wiesen gekommen, die Spatzen im Garten hatten
aufgehört zu schreien. Es dämmerte. Und wenn das
Cello nicht mehr sang, saßen sie still, und er ließ die
Arme sinken und blickte vor sich hin.

Als er die slavische Elegie gespielt hatte, war
es dunkel geworden. Er stellte das Cello beiseite und
lehnte sich an den Flügel. Frau Marie hob langsam
den Kopf und sah ihn an.

Sie stand auf und ging zu ihm hinüber. Die Röcke
rauschten leis. Es war so still im Zimmer.

<l schnee1>Sie standen sich gegenüber, stumm, und die Zeit
sank – wie weicher Schnee sinkt.

Die Frau sagte leis: „Jetzt ist es zu dunkel ge­
worden, um weiter zu spielen.“ Im Garten riefen sich
zwei Mägde zu. Das Geräusch von Schritten drang
in das Zimmer. Dann war es wieder ganz still.

Als er ihr weißes Gesicht sah, das sie ein wenig
in den Nacken zurückgebogen hatte, hoben sich leis
seine Hände. Aber sie blieben stumm sich gegen­
über stehen und horchten wie die Zeit sank – sank.

Und auf einmal gellte die Hausglocke durch die
Stille, des Barons Stimme drang durch den Flur.
Frau Marie trat zurück und drückte auf den elek­
trischen Knopf an der Tür, daß es schamlos hell
wurde, der alte Mann kam, begrüßte Frau Marie,
[s 44]
indem er ihr die Hand küßte, und Hans Jung, in­
dem er sagte: „Es freut mich, daß wir uns schon
wieder treffen.“ Und sie setzten sich zum Tee und
unterhielten sich über Musik und Bücher und Reisen,
bis Hans Jung und der Baron sich gemeinsam ver­
abschiedeten und zusammen durch die Wiesen gingen,
in denen schon die Grillen angefangen hatten mit
ihrem Sommerlärm.[

. . . |
<aa>
]An manchen Abenden saß er in seinem
Zimmer bei Marie vor dem Feldblumenstrauß, zu
dem frische Kornblumen gekommen waren, sie saßen
ohne das Wissen der Mutter, die schlief, und lasen
zusammen von Circe und Eukalypso, den Zau­
berinnen.

Und an den hellen Vormittagen arbeitete er,
holte die große Mappe an seinen Tisch und schrieb
an dem ersten Satz seines Quartetts, versank in
einem tiefen Meer und hörte nicht die Säge des
Bruders, die im Garten hin und her ächzte und
nicht den leisen Gang Maries im unteren Zimmer
und nicht die Spatzen, die durch die offenen Fenster
schrien. Es war schon mittag, wenn er sich in seinen
Stuhl zurücklegte und ruhte. Langsame Schritte kamen
die Treppe herauf, die Stufen knarrten. Er ging zur
Tür und öffnete sie. Es war Marie.

„Ich wollte nur den Rauch ein wenig aus dem
[s 45]
Zimmer ziehen lassen,“ sagte er. Aber es war wenig
Rauch im Zimmer. Marie lächelte und wandte sich,
in ihr Zimmer zu gehen.

„Schauen Sie, so viel habe ich heute schon ge­
schrieben.“ Er zeigte ihr die Notenbögen. Marie trat
zu ihm hin und lachte. Er schloß die Tür hinter ihr.

In dem ersten Fach der Kommode war der
Schmuck, den er aus den Kisten seiner Reiseein­
käufe ausgesucht und mitgenommen hatte. Er nahm
ein zierliches Kettchen von dünnen, braunen Ton­
zylindern heraus, zwischen denen Lotosblumen aus
blauem Porzellan hingen.

„Wollen Sie das?“[|,] sagte er, „es ist aus Indien.“

„Ich?“ Sie nahm das graziöse Ding vorsichtig
in die Hand.

„Es ist nicht viel wert,“ sagte er, „es ist nur
zur Erinnerung.“

Sie dankte ihm und reichte ihm dabei die Hand
wie eine Dame.

„Sie müssen es um den Hals tragen,“ sagte er.
Und er nahm es ihr aus der Hand und streifte ein
wenig ihr Haar, während er es ihr sachte über den
Kopf hob.

„Jetzt sehen Sie aus wie eine indische Prin­
zessin,“ sagte er lachend und sah auf die Lotos­
perle, die zwischen den Brüsten ihres Mieders hing.

[s 46]
„Die Mutter wird nicht erlauben, daß ich es an­
nehme,“ sagte sie und blickte zur Seite, als schäme sie sich.

„Dann zeigen Sie es ihr einfach nicht.“

„Aber wann soll ich es dann tragen?“[|,] fragte
sie lächelnd, und nahm es vorsichtig vom Hals und
betrachtete es.

„Vor dem Spiegel,“ sagte Hans Jung, während
er das Fach wieder zurückschob. „Sie müssen jetzt
oft in den Spiegel sehen.“

Er lächelte noch, als sie schon lange wieder
gegangen war. Er stand am Fenster und sah über
den See und lächelte. Die Bilder hoher weißer
Wolken zogen durch das Wasser.
[
. . |
<aa>
]Wenn er auf der Hotelveranda neben der
Familie mit der Tochter und den ermunternden
Blicken gegessen hatte, und es noch nicht Zeit war,
zu Frau Marie zu gehen, saß er manchmal in der
hellen Wohnstube bei Maries Mutter. Sie bot ihm
Platz an und legte die Bibel in den Schoß und sah
ihn an, als habe er „Mutter“ zu ihr gesagt.

„Ist es nicht langweilig, den ganzen Tag über?“[|,]
fragte sie.

„Mir war es noch nie langweilig,“ erwiderte er.
Es gelang ihm allmählich, mit der Tauben zu reden,
wie ihre Tochter; die Worte leis klingen zu lassen,
wenn sie noch so laut geschrien waren.

[s 47]
„Langeweile kommt erst später,“ sagte sie und
legte die Hände auf die Bibel.

„Wann kommt sie?“[|,] fragte Hans Jung lachend.

Aber er wurde ernst vor dem schweren Ton,
in dem sie sagte: „Es ist verschieden, Marie hat
sie schon jetzt.“ Er schwieg wie vor einer Wunde.

Die alte Frau sah durch das Fenster. „Ich selbst
habe sie erst bekommen,“ sagte sie, „als mein Mann
gestorben war.“

Es wurde ganz still in der Stube. Nur die Uhr
ging hin und her.

„Eines Abends haben sie ihn gebracht,“ sie
hielt ein wenig inne und sagte dann ruhig und ein­
fach: „Der Blitz erschlug ihn auf dem Felde. Es ist
schon lange her.“

„Sie haben doch Ihre Kinder,“ sagte Hans Jung
nach einer Weile.

Sie lächelte. „Ja; und die Bibel.“

„Ja – Gott,“ sagte Hans Jung.

Sie fuhr auf, hastig. „Lügen Sie mich nicht an,“
ihre Stimme klang hart und rauh. „Sie glauben selbst
nicht daran.“

Und dann sagte sie ruhig, und Hans Jung saß
klein vor der einfachen alten Frau: „Aber in der
Bibel stehen Sachen, die sind so schön, daß man sie
immer wieder lesen kann.“

[s 48]
Die Uhr schlug drei. Hans Jung stand auf.

„Ich muß jetzt gehen,“ schrie er. „Ich danke
Ihnen.“ Aber er wußte nicht, wofür.[

. . |
<aa>
]Einmal, als er in dem Erkerzimmer ein wenig
warten mußte und am Fenster stand und in den
Garten sah, ging die Tür auf und Doktor Blank
kam auf ihn zu.

„Sieh da, sieh da, unser Weltreisender,“ sagte
er in dem lauten Ton, in dem er schon immer ge­
sprochen hatte. Er reichte Hans Jung beide Hände,
und ehe dieser noch dazu kam, ihn zu begrüßen,
erzählte er, daß er <l uniferien>zwei Tage Universitätsferien be­
nutzt habe, um seine Frau zu überraschen, und daß
er sich freue, ihn bei dieser Gelegenheit endlich
wieder einmal zu Gesicht zu bekommen.

„Sie sehen gut aus, Herr Doktor,“ sagte Hans
Jung und stand wieder von seinem Stuhl auf, um
Frau Marie zu begrüßen, die hereinkam. Er ver­
beugte sich über ihre Hand, ohne ihr in die Augen
zu sehen. Sie setzten sich zusammen in den Erker.

„Jetzt war ich lang genug Strohwitwer,“ sagte
der Doktor lachend und bot Hans Jung eine dicke
Zigarre an. Hans Jung dankte. „Haben Sie sich
nicht eine kleine Chinesin mitgebracht?“[|,] fuhr er fort.
Seine Gedankenverbindungen waren schon immer
sehr offensichtlich gewesen.

[s 49]
Hans Jung lachte. „Ich glaube, sie würde Europa
nicht gut vertragen,“ sagte er.

„Ja, es soll ja in China eine hohe Kultur sein,“
sagte der Doktor und zog seine Stirn in quere Falten
und stützte sich schwer auf den Tisch.

Frau Marie sagte sehr ruhig: „Davon können
wir uns wohl kaum eine Vorstellung machen.“

Hans Jung sah sie einen Augenblick an, während
der Doktor den Rauch seiner Zigarre von sich blies
und sagte: „Sie müßten ein Buch über Ihre Reise
schreiben.“

„Warum?“[|,] fragte Hans Jung und lächelte.

„Warum?“[|,] sagte der Doktor und sog an der
Zigarre, ehe er weiter redete: „Na, wissen Sie, wenn
Sie auch kein Geld damit verdienen wollen, könnten
Sie doch gewiß Neues bringen, Interessantes und
so weiter. Wenn ein jeder still sein wollte nach
einer so langen Reise, gäbe es überhaupt keine Reise­
beschreibungen mehr.“

„Das wäre ja herrlich,“ sagte Hans Jung; aber
als er sah, daß Frau Marie nicht mit ihm lächelte,
fuhr er leiser fort: „Man muß auch für sich behalten
können. Man muß sehr viel für sich allein behalten
können.“

„Gäbe es dann nicht recht wenige Künstler?“ Frau
Marie sprach so ruhig, daß er den Kopf senken mußte.

[s 50]
„Geformt – das ist etwas anderes,“ sagte er,
und es fielen ihm nur mühsam die rechten Worte
ein. „Nur nicht photographiert.“

„Aber die Grenzen verwischen sich,“ erwiderte
sie, ohne ihn anzusehen.

Der Doktor, der immer ein wenig ärgerlich wurde,
wenn sich das Gespräch ins Begriff·liche wandte, sagte:
„Unser Jünglein ist gewiß nur zu stolz.“

„Das kann sein,“ sagte Hans Jung lachend,
während das Dienstmädchen hereintrat und die gnädige
Frau zu dem Diener des Barons bat.

„Er soll herüberkommen,“ rief ihr der Doktor
nach.

„Kennen Sie Mannen?“[|,] fragte er Hans Jung,
als sie gegangen war. „Ist das nicht ein komischer
Kauz?“ Er schob die Bauchbinde von seiner Zigarre
und sagte gemütlich: „Na ja, ein jedes Tierchen hat
sein Pläsierchen und besonders ein alter Junggeselle.
Aber er hat doch seiner Lebtag lang ein zu eigen­
artiges Pläsierchen gepflegt.“

„Was?“[|,] fragte Hans Jung und hörte im Gang
draußen Frau Maries Stimme, ohne sie verstehen
zu können.

„Was?“[|,] sagte der Doktor, „Sie können es mir
glauben oder nicht: er hat in seinem ganzen Leben
noch kein Weib angerührt.“

[s 51]
Hans Jung antwortete nichts.

„Na, Sie verstehen, wie ich es meine – das
nenne ich eine seltsame Philosophie, was?“

„Erzählt er es selbst?“[|,] fragte Hans Jung.

„Ich weiß es gewiß,“ sagte der Doktor und
lachte – – und Hans Jung wußte nicht, warum
ihm auf einmal traurig wurde, wie seit dem Frühling
nimmer.

„Wissen Sie,“ fuhr der Doktor fort, „ich lasse es
mir gefallen, wenn einer durch seine Philosophie ein
wenig verschroben wird. Aber solche Theorien sind
einfach krankhaft. Der Beischlaf ist doch ein Akt –“

Frau Marie kam zurück und sagte: „Er wird
gleich herüberkommen.“

Der Doktor begann von der Musik, ob sie Hans
Jung noch fest betreibe und von Fritz, ob der nicht
ein Prachtbursch sei, und wie die Zeiten kämen und
gingen.

Später kam der Baron, und sie tranken gemeinsam
Tee. Es gab eine flotte Unterhaltung. Der Doktor
fühlte sich als Mittelpunkt und Gastgeber, er erzählte
von der verflucht vielen Arbeit in der Klinik, und
daß er wenig Zeit mehr fände, sich der Kunst zu
widmen, was er gerne möchte, und ob Fritz nicht ein
Prachtbursch sei.

„Sing' uns was vor, Marie,“ sagte er gemütlich.
[s 52]
Der Baron und Hans Jung schwiegen und sahen vor
sich hin wie in einem stillen Einverständnis.

Die Frau erhob sich wie ein williges Kind und
setzte sich an den Flügel. Sie suchte ein wenig in
den Noten, aber dann legte sie sie wieder beiseite
und spielte aus dem Kopf. Sie sang ein kleines Lied,
das Hans Jung nicht kannte.

Eine kleine chromatische Tonleiter, die jäh in
der Höhe abbrach, war das Vorspiel. Und dann
kamen drei kurze, strophisch komponierte Verse mit
nur wenig Melodie, nichts wie ein breites Heben und
Senken, ein wenig sentimental an manchen Stellen.
Sie sang langsam und leis; es klang fast als phantasiere
sie nur für sich . . . Ein alter Mann stand an seinem
Fenster und sah hinaus, einmal in die Dämmerung
und einmal in einen langen Regen und einmal in seinen
Garten, in dem kleine Mädchen spielten. Er stand
und wartete – wartete nicht auf eine Frau und nicht
auf den Tod und nicht auf Wunder. Er stand und
wartete, wartete auf nichts. Die letzte Strophe endete
wie eine klagende Frage.

„Bravo,“ sagte der Doktor, als sie schwieg.

Hans Jung sah den Baron an. Er hatte den
Kopf während des Liedes gewandt und blickte in
den Garten, auf den Baum vor dem Fenster, es sah
aus, als höre er nicht auf Frau Mariens Stimme.

[s 53]
„Das ist ein seltsames Lied,“ sagte Hans Jung
und wandte sich allein an ihn, „kannten Sie es schon?“

„Nur den Text,“ sagte der Baron.

„Der Text ist auch das beste daran,“ sagte Hans
Jung. „Die Komposition ist ein wenig zu dilettantisch.“

„Sie haben recht,“ sagte Frau Marie, die noch
am Flügel saß und zuhörte. Und Hans Jung sah,
wie sie die Augen des alten Mannes suchte.

„Jetzt noch etwas Frisches, bitte, Marie,“ äußerte
der Doktor. Und sie sang ein kleines russisches Lied,
das alle kannten: „Nachtigall, ach Nachtigall . . .“

Der Baron stand auf, um sich zu verabschieden.
Hans Jung schloß sich ihm an, und der Doktor machte
nur geringe Anstrengungen, sie zurückzuhalten.

„Wollen Sie nicht mein Haus ansehen“, sagte
der Baron, als sie über die Wiese gingen, „und
vielleicht bei mir zu Abend essen? Es ist doch ein
angebrochener Nachmittag.“

Hans Jung war überrascht und nahm gerne an.

„Den Abend kann ich Ihnen leider nicht widmen,“
sagte der Alte, und es fiel Hans Jung auf einmal auf,
wie rüstig und jugendlich er seine Schritte setzte.
„Am Abend muß ich arbeiten.“

Aber er sagte nicht, was er arbeitete, und Hans
Jung wagte nicht zu fragen.[
|
<aa>]
Es war eine der letzten alleinstehenden Villen,
[s 54]
ein paar Schritte neben der Hauptstraße. Das Zimmer,
in das der Baron ihn führte, war groß und kühl,
mit nur wenigen breiten Eichenmöbeln. Sie saßen
an einem kleinen Tisch am Fenster, von dem aus
man auf die Wiese hinter dem Haus sah und auf
den Wald, der sie abschloß und zum Berg anstieg.
Schwere schwarze Wolken zogen über die Grate.

„Es ist schwül,“ sagte der Baron, „wir werden
ein Gewitter bekommen.“

Ein junger, braungebrannter Diener in dunkel­
grünem Dreß brachte Wein, der in kleinen Eiskübeln
kühlte und kredenzte ihn in flachen Schalen. Er
bediente mit jenem stummen und fast monumentalen
Zeremoniell, wie es Hans Jung nur ein paarmal in
Rußland gesehen hatte. Als er eingeschenkt hatte,
blieb er steif hinter seinem Herrn stehen und wartete,
bis er ihm winkte, zu gehen. Es wirkte wie eine
groteske Laune des Alten.

Hans Jung hatte das Verlangen, dem alten Manne
angenehm zu sein. Der Baron hatte den Kopf gewandt
und blickte auf die Gewitterwolken hinaus – mit
ganz anderem Ausdruck in seinen Zügen, als es Hans
Jung bei ihrem ersten Zusammentreffen gesehen hatte.

Hans Jung begann von seinen Reisen zu erzählen.
– Es war schon lange her, daß er den letzten Wein
getrunken hatte. Wo war es gewesen? – In Griechen­
[s 55]
land. In einem entlegenen Dorf, auf dem Peloponnes.
In dem einzigen Xenodeichion des Dorfes war er ge­
sessen, in einer Laube, die blühte, und hatte auf die
Wolken hinausgesehen, die schwarzer und schwerer
wie jene da drüben über die blauen Waldberge zogen.
Die Tochter des Wirts war gekommen und hatte
einen Krug kühlen Wein gebracht und sich lachend
zu ihm gesetzt und ein Lied gesungen, dessen Sinn
er nicht verstand . . Der alte Mann horchte und
schwieg.

Sie war groß und schlank gewesen, mit hohen
Hüften und tiefschwarzem Haar und straffen Brüsten.
Und sie hatte ein Kleid getragen, das rostfarben war
wie die Segel der griechischen Schiffe. Aber das
Lied hatte viele, viele Strophen auf die gleiche kleine
Melodie, und er hatte nur immer den Klang der
köstlichen fremden Worte gehört.

„Es war gewiß ein Königslied,“ sagte der Baron,
als Hans Jung schwieg. „Ich hätte es vielleicht gekannt,
ich lebte einige Jahre in Athen.“

Er ging zu einer hohen Truhe, die in der Ecke
stand und nahm eine Mappe aus ihr.

„Ich zeichnete damals ein wenig,“ sagte er,
„vielleicht ist manches dabei, das Sie kennen.“

<s>Es waren einige einfache Skizzen; halbverwischte
Kohlenzeichnungen und ein paar helle Aquarelle.
[s 56]
Er blätterte in der Mappe, ohne Rücksicht auf Hans
Jung; es war, als wolle er nur sich selbst erinnern . . .
ein paar Dorfstraßen, eine Amphore, eine alte Frau,
die auf einem Stock durch einen zerfallenen Tempel
ging; und dann ein Blick durch eine Säulenreihe auf
das Meer.

„Das Parthenon,“ sagte Hans Jung.

„Haben Sie schon etwas Schöneres gesehen, als
die Akropolis?“[|,] fragte der alte Mann, während er
die Mappe wieder in die Truhe zurücklegte.

Hans Jung sah auf die magere Greisenhand,
deren Finger leis mit dem Schaft der Weinschale
spielten. Blaue Adern liefen darauf. Die Worte
Doktor Blanks fielen ihm plötzlich ein, während er
die alte Hand betrachtete.

„Die Frauen auf dem Balkan,“ sagte er, nach­
dem sie ein wenig geschwiegen hatten, „haben alle
hohe schlanke Hüften. Und manche habe ich zum
Brunnen gehen sehen wie Araberinnen.“

„Ich glaube, es regnet schon,“ sagte der Baron
und sah unverwandt durch das Fenster.

<l ägypten1_anfang>„Aber die Araberinnen sind doch die schönsten
Frauen, die ich kenne,“ fuhr Hans Jung fort; aber
als er merkte, daß der Alte nicht auf ihn achtete,
lachte er ein wenig auf, und es war kein Wort wahr
von dem was er sagte: „Die Araberinnen! – Ich
[s 57]
habe einen Freund, der sehr lange in Ägypten war,
und der immer sagte, es sei ihm ein unheimliches
Gefühl, nicht zu wissen, ob seine Kinder braun oder
weiß geworden sind.“

Die alte Hand auf dem Tisch hob sich ein wenig
und senkte sich wieder.

„Ist Ihr Freund darum nach Ägypten?“[|,] fragte
der Baron.

„Er ging hin, um die Wüste zu malen,“ sagte
Hans Jung. Und er schwieg ein wenig und senkte
den Kopf traurig für den Freund, ehe er sagte: „Aber
es gelang ihm nicht. <l spannungsfeld>Und er wäre vielleicht trostlos,
wenn er nicht jene Lust an Körperlichkeit hätte, zu
der er sich flüchten konnte. Glauben Sie nicht?“

„Nein,“ sagte der Baron hart, „aber ich glaube es,
daß er die Wüste nicht malen konnte, wenn er nicht
einsam genug war.“ Er hatte die Hand ein wenig
von sich geschoben; es schien Hans Jung eine Be­
wegung, als wolle er sich vor etwas Unreinem schützen.

„Aber ich glaube,“ sagte Hans Jung ein wenig
leiser, „daß ein Künstler alles schaffen kann, auch
das einsamste, wenn er auch von einer Nacht kommt,
in der er mit einer Frau schlief.“

Der Baron hob langsam das feingeschliffene Glas
zu den dünnen Lippen. „Hören Sie,“ sagte er, „nun
wird das Gewitter bald bei uns sein.“

[s 58]
Leise lange Donner tönten aus den Schluchten.
[|<a>]
Als es Abend wurde, gingen sie in das Speise­
zimmer hinüber und aßen fast schweigend. Der steife
Diener servierte.

„Was ist das für ein köstlicher Kopf?“[|,] fragte
Hans Jung. Es war ein ägyptischer Basaltkopf, am
Hals abgebrochen, der auf einer Kommode neben
dem Tisch stand. Der Alte holte ihn herüber und
betrachtete ihn mit Hans Jung zusammen.

„Dieser Mund,“ sagte Hans Jung, „das ist das
Schönste an den ägyptischen Köpfen.“

Der Alte lächelte und stellte den Kopf wieder
auf seinen Platz zurück.

„Sie haben recht,“ sagte er. „Und so wenige
Menschen wissen etwas vom Mund.“ Und erst eine
Zeitlang später, als sie fertig gegessen hatten und
sich in ihren Stühlen zurücklehnten, fuhr er fort, leise
lächelnd: „Sie – und Ihr Freund werden vom Mund
sagen, er ist nur für die Frauen da.“

„Aber die Ägypter,“ fuhr er fort, „haben es
ein wenig besser gewußt als die meisten unserer
Künstler.“

„Was wußten die Ägypter?“[|,] fragte Hans Jung
und wagte nicht den Kopf zu heben.

Der Alte sagte: „An dem Mund dort ist das
Schönste seine göttliche Verachtung.“ Und er sprach
[s 59]
auf einmal in einem so sonderbaren Ton, daß Hans
Jungs Stimme versagt hätte, zu antworten.

„Wenn die Ägypter ihre Philosophie nicht in
den Mund ihrer Götter gelegt hätten, und sie hätten
niederschreiben müssen, wäre es vielleicht eine Philo­
sophie des Stolzes oder der Verachtung geworden. . .
Ich glaube, sie haben gewußt, daß wir nur schaffen,
um verachten zu können. . . Wir sagen, wir werden
reicher – aber es ist nur, daß wir stolzer werden
und mehr verachten können. . . .“

Hans Jung schwieg und sah zu dem Kopf mit
dem abgebrochenen Hals hinüber.<l ägypten1_ende>

Aber als er später sprach, war ein Klang in
seiner Stimme, als wolle er die alte, dünne Hand
küssen.[ –
|<a>]
Frische Blumen waren in dem Krug auf seinem
Tisch. Er trat ans Fenster und öffnete es weit. Es
war eine große Stille rings, wenn auch der Donner
immer näher kam. Hans Jung stand stumm und
bewegungslos am Fenster. Später schoß ein greller
Blitz vergehend in den See, und dann kam ein
Donner, daß die Scheiben klirrten. Es war ganz
dunkel geworden.

Es klopfte leise an die Tür, Marie kam, um
frisches Wasser zu bringen.

„Sie haben noch kein Licht,“ sagte sie.

[s 60]
„Nein.“ Er rührte sich nicht.

Sie fand die Krüge auch im Dunkeln und ging
still hin und her.

Als sie den Krug mit den Blumen zurückbrachte,
hatte er die Lampe angezündet. Das Unwetter war
vorübergezogen, eine köstliche Luft kam durch die
offenen Fenster.

„Die Mutter und Thomas sind schon zu Bett,“
sagte sie, als sie die Blumen auf den Tisch gestellt
hatte.

„Haben Sie sich nicht vor dem Gewitter ge­
fürchtet?“[|,] fragte Hans Jung und lehnte sich neben
sie an den Tisch.

„Nein,“ sagte sie ernst.

„Was fürchten  S i e ?“[|,] fragte er leis.

Sie schwieg ein wenig. Der Donner kam schwach
aus den Bergen zurück.

„Nichts,“ sagte sie. „Gute Nacht.“

„Schon schlafen gehen,“ sagte Hans Jung.
„Wollen wir nicht noch ein wenig zusammen lesen?“

Sie besann sich einen Augenblick.

„Ich muß erst das Licht in meinem Zimmer
auslöschen,“ sagte sie dann. „Ich komme gleich
zurück.“

Hans Jung stand am Tisch, regungslos, und
wartete.

[s 61]
Und dann saßen sie unter dem gelben Schein
der Lampe, das Zimmer rings war im Dunkeln, nur
die Decken des Bettes leuchteten heraus. Es hatte
aufgehört zu regnen. Wenn sie schwiegen, war es
still im ganzen Haus.

Sie lasen nicht. Er erzählte ihr nur. Er er­
zählte von seinen Reisen, von Japan und vom Meer
und von den schillernden Südseeinseln. Und er
bildete kurze, klingende Sätze, daß die Farben aus
ihnen leuchteten. Von den chinesischen Straßen mit
ihrer Musik, die niemand vergißt, der sie einmal
gehört hat; von ihren tausend Papierlaternen, aus
denen die Drachen leuchten . . . von den fremden
Frauen, die sich mit seltsamen Ketten schmücken
und sich in Sänften tragen lassen, die mit bunten
Seiden beschlagen sind.

„Wollen Sie gern einmal dorthin?“[|,] fragte er
lächelnd, während ihre Augen unverhüllter an seinen
Lippen hingen.

Er holte das kleine schwarze Heft und las ihr
ein paar Gedichte vor, die ihm ein junger chinesischer
Dichter übersetzt hatte, mit dem er in Shanghai
kurze Zeit befreundet gewesen war. Es waren
uralte Lieder aus dem Schi=King. Von Frauen,
deren Brüste bebten, während ihre Männer im Krieg
waren; von Wolken und von Blüten . . und ein Lied,
[s 62]
in dem sich zwei Menschen für immer liebten, nach­
dem sie sich nur einmal unter einem jungen Baum
kurz in die Augen gesehen hatten.

„Ich muß jetzt gehen,“ sagte sie, als er das
Heft schloß. „Gute Nacht.“

Sie stand auf, zögernd.

„Jetzt hat es aufgehört zu regnen,“ sagte Hans
Jung und trat ans Fenster zurück. „Was ist das für
ein Licht dort drüben?“

Sie trat aus dem Schein der Lampe und stellte
sich neben ihn.

„Das ist ein Haus über dem See.“

Sie sahen still auf die Schatten des Wassers hinab.

Hans Jung hob langsam die Hände und faßte
sie um den Hals. Er spürte das Blut in ihren Adern
klopfen. Aber sie stand stumm und regungslos und
wartete. Er küßte sie. Da lehnte sie sich an ihn,
schwer und traurig.

„Gute Nacht,“ flüsterte sie.

Er küßte ihr Haar und zog sie an sich. Wie still es
war im Haus . . . sie wehrte nicht seinen leisen Händen.

Und ihre kleinen Brüste dufteten wie seltene
Blumen.

Als der erste Schein der Dämmerung kam,
entglitt sie ihm. Sie nahm leis ihre Kleider vom
Boden und schlich in ihr Zimmer zurück.

[s 63]
[. . |<aa>
]Als er wieder erwachte, war es heller Tag.
Er sprang aus dem Bett, daß die Decken zu Boden
fielen. Die Spatzen schrien im Garten. Er trat ans
Fenster, nackt und straff. Die ersten Strahlen kamen
flach über die Berge. Das Wasser brannte in tausend
Flammen. Der Himmel war klar und glockenblau.
Er riß die Scheiben zurück, daß sie klirrten. . .
Irgendwoher kam ein Klingen von Frau Maries
Namen. Da stand er mitten im hellen Raum und
warf die Arme in die Höhe, jauchzend.

Dann stand er stumm vor dem Bett.

[s 64]

