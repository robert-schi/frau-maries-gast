<k Dritter Teil.>


Hans Jung erwachte, ehe es Morgen war. Noch
waren die Vögel still hinter dem Haus. Als sie
später anfingen in den Sträuchen zu singen, erst
einzeln, dann mit dem ganzen Lärm ihrer Antworten,
sprang er aus dem Bett. Ehe er sich ankleidete, trat
er nackt ans Fenster und sah auf den See, auf den
die Streifen der Morgensonne gekommen waren.
Am untern Ende trieb ein Boot mit einem Mann.
Er stand wie ein schwarzer Strich gegen das glänzende
Wasser und ließ Boot und Ruder gleiten. Es schien,
als rühre er sich nicht. Eine kleine weiße Wolke
trieb einsam über ihm, mitten in dem leuchtenden Blau.

Maries Bruder kam aus dem Haus und ging
zu seinem Boot hinab, das an einem Pflock hing und
halb aufs Land gezogen war. Er pfiff immerzu das
gleiche Stück aus einem kleinen Ländler. Sein schnee­
weißes Hemd erinnerte Hans Jung daran, daß es
Sonntag war. Da sah auf einmal auch der See
und der Himmel feiertägig aus.

[s 111]
„Marie!“[|,] rief der Bruder, als er das Boot los­
gebunden hatte, „Marie!“

Marie kam langsam über den kleinen Rasen.
Sie trug ein Kleid, das Hans Jung noch nicht bei
ihr gesehen hatte. Ein schwarzer Rock, der in dicke
lange Falten fiel und ein schwarzes Mieder, das ihre
Brüste nicht in die Höhe drängte. Und ein rotes
seidenes Tuch, das sie um die Schultern trug, als
friere sie. So frauenhaft sah sie aus in der unge­
wohnten Tracht.

Sie stieg in das Boot, das ihr der Bruder hielt
und saß stumm auf der kleinen Bank am Kiel, während
er es ins Wasser schob und nachsprang, daß die
Planken krachten. Er streifte das Hemd über die
braunen Arme hinauf und begann zu rudern.

Als sie ein wenig vom Ufer entfernt waren,
hob das Mädchen plötzlich den Kopf und sah unver­
mittelt zu Hans Jungs Fenster empor. Er war schnell
zur Seite getreten. Aber als er wieder hinuntersah,
vorsichtig vom Fenstereck aus, hatte sie den Kopf
gewandt und sah ins Wasser, das kleine Wellen um
das Boot warf. Das Tuch war von ihren Schultern
geglitten.

Der Bruder begann ein eintöniges Lied und sang
bei jedem Ruderschlag eine Strophe.

Hans Jung trat vom Fenster zurück. Er kleidete
[s 112]
sich sorgfältig an und ging hinter das Haus. Ruhig
und still, wie ein alter Mann, saß er auf der Bank
im Rasen, während die Sonne immer höher ging und
die Glocken im Dorf zu läuten begannen. Das Haus
blieb stumm; vielleicht schlief die Mutter noch. Manch­
mal glucksten ein paar kleine Wellen auf, daß er
Lust bekam, in dem klaren Wasser zu schwimmen.
Aber er blieb versonnen sitzen und sah vor sich hin.

Er hörte das Boot erst zurückkommen, als es
schon nahe am Ufer war. Er ging ihnen entgegen
und gab ihr die Hand, als sie aufs Land sprang.

„Zeit ist's für die Kirchen,“ sagte der Bruder,
während er das Boot anband. Hans Jung stand mit
Marie hinter ihm und half ihr das Tuch auf die
Schultern zurücklegen. Sie dankte ihm mit einem
kleinen, ein wenig verängsteten Lächeln, daß er sie
gerne um die Brust genommen hätte, mitten in der
Sonne, die ihren Körper überfloß.

„Gehen Sie in die Kirche?“[|,] fragte er.

„Ja,“ sagte sie, und das Blut stieg ihr auf einmal
zu Gesicht, daß sie sich wenden mußte.

Hans Jung ging in sein Zimmer zurück.

Er holte die Notenmappe vor und setzte sich
an den Tisch, um zu schaffen. Aber er legte die
Blätter bald wieder zur Seite. Es waren keine Tage,
zu arbeiten. Es waren Tage, lange auf den See
[s 113]
hinauszusehen, der voller Glast war, und durch den
Wald zu gehen, in dem die Düfte hingen; und bei
der Frau zu sitzen, still.

Marie kam, um das Zimmer zu ordnen.

„Sind Sie nicht in die Kirche?“[|,] fragte er.

„Nein,“ sagte sie und blieb vor ihm stehen,
ohne sich zu rühren. „Ich gehe erst am Nachmittag.
Die Mutter und Thomas sind gegangen.“

„Dann ist das Haus leer,“ sagte er lächelnd –
und er war auf einmal so froh, daß er sie hatte,
um ohne seine eigenen Gedanken sein zu können.

Als er sie an sich zog, sank ihr Kopf schwer
auf seine Brust, wie bei einem Kind, das nach langem
Weinen ruhig geworden ist.

„Wir sind unsere eigene Kirche, Marie,“ sagte
er leis.

Sie nahm ihn nicht mehr scheu und knospend
auf, – bot ihm die Lippen wie eine Frau, die voller
Sommerblüten ist.

„Die Mutter wird bald zurück sein,“ sagte sie,
als sie sich ihm entzog.

Nun war er wieder allein. Die Sonne schien
durch die Fenster, daß das ganze Zimmer voll
davon war.

Ehe er ging, nahm er aus der kleinen Schatulle,
die auf der Kommode stand, den kostbarsten Schmuck,
[s 114]
den er von seinen Reisen mitgebracht hatte. Er hüllte
ihn in ein seidenes Tuch, um ihn zu sich zu stecken.

Es war eine Armkette von großen chinesischen
Perlen, zwischen denen geschnittene Jadestäbchen
hingen.

Wenige Leute begegneten ihm auf der Dorf­
straße. Aus der Kirche tönte Gesang und Orgelspiel.
Er besann sich einen Augenblick hineinzugehen. Aber
dann wandte er sich und war auf einmal, ohne es
recht zu wissen, in der Wiese, an der ihr Haus lag.
Es war noch zu früh zu einem Besuch; er durfte
nur vorübergehen. Der Bub spielte im Garten.
„Komm, Fritz!“[|,] rief Hans Jung vom Tor aus. Er
kam und gab ihm zögernd die Hand durch die Stäbe
des Gartentores, das geschlossen war. Wie hell es
um den alten Nußbaum war . . Die Fenster ihres
Schlafzimmers waren geöffnet; das Dienstmädchen
ging hin und her darin. Hans Jung wandte sich und
ging den Wiesenweg zurück. Der Bub drückte die
Stirn an das Gitter und sah dem Fremden nach, bis
er verschwunden war.

Die Hauptstraße war voll Bauern und Bäue­
rinnen. Die Kirche war aus, und die Glocken läuteten
immerzu. Hans Jung ging ein Stück den See ent­
lang, auf dem Promenadeweg der Hotelgäste. Er
setzte sich auf eine Bank zwischen zwei alten Bäumen
[s 115]
mitten in der Sonne. Es war angenehm, still nach
den Streifen des Sees zu sehen, die durch die Bäume
blinkten und nach den Sonnenflecken in den Büschen
und nach den Herren und Damen, die vorüber­
gingen. Wie man ihre tiefsten Herzensangelegen­
heiten kannte, wenn man sie nur kurz gesehen hatte.

Das Ehepaar vom Hotel war unter ihnen, mit
der Tochter. Sie lief hinter den Eltern und pflückte
Blumen und sprang über den Graben, der kurz vor
Hans Jungs Bank war. „Wie ist Mariechen über­
mütig,“ sagte die Mutter sehr laut zum Vater, der
eine riesige Morgenzigarre rauchte. Ach, Mariechen
hieß sie! Mariechen trug ein festliches Sonntagskleid,
das sich ihren runden Formen aufs äußerste an­
schmiegte.

Und dann kamen wieder die zwei Jungver­
heirateten, schüchtern wie immer, vielleicht weil sie
fühlten, daß man ihnen von weitem ansah, wie gern
sie sich hatten. Sie gingen Hand in Hand, und Hans
Jung hätte ihnen dafür gern Blumen in ihr kleines
Hotelzimmer gestellt.

Aber auf einmal stand vor seiner Bank ein
junger Mann, der durch die Bäume vom See her
gekommen war und den Hut in der Hand trug. „Du
bist der Hans Jung,“ sagte er und betrachtete ihn
überrascht; es war ein Kamerad aus der Gym­
[s 116]
nasiumszeit. „Lebst du noch?“[|,] fragte er, während
sich Hans Jung auf seinen Namen besann, der ihm
entfallen war. Der Schulfreund trug einen grünen
Lodenanzug und war nur auf einen Tag am See;
er befand sich auf einer großen Fußtour. Und wie
es Hans Jung ginge, und ob sie zusammen zu Mittag
essen wollten, und ob er noch viel Musik treibe.
Sie sprachen vom Gymnasium, von den Mittel­
schülern und von den Mädchen, die sie gemeinsam
gekannt hatten. Es war Hans Jung nicht ungelegen,
die Wartezeit bei Dingen zu verbringen die ihm
klangen, als habe er sie vor langer Zeit einmal in
einem Buche gelesen. „Ich bin verlobt, Hans,“ sagte
der Schulfreund, als sie beim Mittagessen auf der
Veranda des Hotels saßen. Hans Jung gratulierte.
„Wir wollen im Herbst Hochzeit machen,“ sagte der
andere. Hans Jung wußte noch immer nicht seinen
Namen; aber er wollte ihn nicht darnach fragen.
„Messerschmidt“ oder „Schildhammer“ – ähnliches
fiel ihm immer ein. – „Weißt du,“ sagte der Schul­
freund und zog die Stirn in Längsfalten, die er früher
nicht gehabt hatte, „schließlich sind wir nicht auf der
Welt, um Junggesellen zu bleiben. Du wirst auch
noch daran glauben müssen.“ Er sprach von den
vielen, vielen Annehmlichkeiten, welche die Ehe mit
sich bringt. „Man spricht nicht gern davon,“ sagte
[s 117]
er, „aber ich kann es ruhig sagen: wir sind sehr
verliebt ineinander. Und meine Braut ist ein liebes
Mädel.“ Hans Jung fühlte sich sehr gemütlich; er
beobachtete nebenbei, wie Fräulein Mariechen am
Nebentisch sich Mühe gab, graziös zu essen, er freute
sich ein wenig über die Unsicherheit, die den andern
allmählich von selbst überkam, und wie es ihm nur
schlecht gelang, vornehm zu wirken, indem er barsch
mit den Kellnern sprach. Er war Assessor geworden,
wie sich herausstellte und hatte ein gutes Examen
und viel Arbeit hinter sich. „Und von dir erzählst
du gar nichts,“ sagte er. „Was hast du alles erlebt
die Zeit über?“ – „Ach,“ sagte Hans Jung, „ich
habe mich ein wenig herumgetrieben und musiziert
und dergleichen.“ – „Und dich plagen müssen, deine
Zinsen auszugeben,“ sagte der Assessor und lachte.
Hans Jung erwiderte nichts und sah ein wenig nach
der Brust des Touristenhemdes, das nicht mehr ganz
frisch war. Nach dem Essen mußte er sich ver­
abschieden; er konnte das freudige Wiedersehen leider
nicht weiter feiern, wie der Assessor gern gewollt
hätte. Aber es war ihm noch nicht eingefallen, wie
er hieß, als sie sich schon getrennt hatten.

Hans Jung ging über die Wiesen, das Cello
unter dem Arm.

Frau Marie saß in der Laube hinter dem Haus
[s 118]
und hatte Gäste zum Kaffee: Baron Mannen, der
Hans Jungs Hand heute fest drückte, als er ihn
begrüßte; ein kleiner verwachsener Herr, der Doktor
Norten hieß, und dessen Kopf wie der eines alten
Kindes nur wenig über die Tischplatte ragte; und
seine Schwester, ein junges Mädchen in hellem
Sommerkleid; Berta Norten hieß sie.

Aber Frau Marie trug ein Kleid aus leichter
grauer Seide, das ihren Hals und Nacken frei ließ
und sah Hans Jung bei seinem Kommen an, daß
ihm war, die Vögel im Garten müßten vor Lust
stocken, zu singen.

Der Baron sagte: „Sie müssen den Sonntag
fertig machen, Herr Jung und später Cello spielen.“
Und Doktor Norten fügte hinzu: „Mannen hat mir
schon von dem Cello erzählt.“ Und er sah mit
seinen stahlblauen Augen zu ihm hinauf. Aber das
junge Mädchen schaute Frau Marie an, die ihn leis
fragte, wie es ihm gehe.

Hans Jung setzte sich und sah lächelnd auf die
Sonnenflecken, die durch die Lücken des Strauch­
werks kamen und sich auf der dunkelgrünen Decke
lagerten.

Sie waren in einem Gespräch über die Hand­
werksburschen gewesen. „Sie haben gewiß auf Ihren
Reisen schon sehr viele Handwerksburschen gesehen,“
[s 119]
sagte der Baron. Und der kleine Doktor sagte, es
klang fast wie eine Entschuldigung, wenn er sprach:
„Ich habe nämlich gestern auf einer Bank im Wald
einen struppigen alten Mann getroffen. Er bettelte
mich an, und ich unterhielt mich ein wenig mit ihm.
Und er behauptete, schon seit vierzig Jahren auf
der Wanderschaft zu sein, ohne Arbeit, ohne Ziel,
ohne Freunde. Und doch machte er nicht den Ein­
druck, als ob er ganz wertlos sei.“

„Warum soll er wertlos sein?“[|,] fragte Hans Jung
und lächelte.

„Darüber läßt sich doch reden,“ sagte der
Doktor, „erstens leistet er in seinem ganzen Leben
nichts für die menschliche Gesellschaft.“

Der Baron unterbrach ihn. „Das tue ich auch
nicht,“ sagte er und lachte hart auf, und Doktor
Norten wurde rot wie ein Mädchen.

„Sie wissen, wie ich es meinte, Mannen,“ sagte
er. „Aber vor allem lebt er immer von fremdem
zusammengebettelten Geld, und das finde ich schuftig.“

„Das finde ich nur lustig,“ sagte der Baron, der
heute jünger schien als je.

„Ob lustig oder traurig,“ erwiderte der Kleine,
„auf jeden Fall zeigt es sehr wenig Stolz.“

„Oder sehr viel Stolz,“ sagte Baron Mannen.
Es schien fast, als wolle er ihn necken. Aber er
[s 120]
fuhr in ernstem Tone fort: „Er kann ja so stolz
sein, daß es für ihn durch die Menschen, die er
anbettelt, keine Demütigung gibt.“

Doktor Norten senkte den Kopf ein wenig.
„Es kann sein,“ sagte er leis.

Sie schwiegen, bis Hans Jung, der seine Hand
auf einmal auf die sonnige Decke gelegt hatte, dicht
neben die Hand der Frau, zu sprechen begann, und
seine Stimme klang ein wenig rauh: „Ich war sogar
einmal mit einem Handwerksburschen befreundet,“
sagte er, „es war ein Deutscher, der zuerst Student
gewesen war und dann ohne Geld ganz Europa
durchwandert hatte. <l ägypten2_anfang>Ich traf ihn in Ägypten.“ Sie
horchten, wie wenn jemand aus einem Buch vor­
liest. „Es war auf einem Nilschiff, wo ich ihn traf.
Er machte einen Steward; um nach dem Sudan zu
kommen, aber er blieb dann einige Wochen als Gast
bei mir, bis er eines Tages ohne Abschied ver­
schwunden war. Ich hatte es nicht anders erwartet,
und ich weiß nicht, wohin er gegangen ist. Es war
einer der wertvollsten Menschen, die ich gekannt
habe.“

„Wieso?“[|,] fragte das junge Mädchen.

„Wieso?“ Hans Jung lächelte. „Er brauchte
keine Freunde und keine Bücher und keine Musik,
um stolz zu sein und den Kopf zu tragen wie ein
[s 121]
junger König. Und er brauchte nicht die Sicherheit,
etwas zu leisten, oder etwas geleistet zu haben, um
sich wertvoller zu fühlen als die andern. Er  w a r 
es einfach. Es gibt natürlich wenige solche unter den
Handwerksburschen; die meisten sind wirklich ver­
kommen. Aber ich kannte noch einen solchen,“ er
wandte sich zu Frau Marie, „das war der Nord­
länder am gelben Fluß.“

„Ja,“ sagte sie; es war wie wenn sich ein Vogel
duckt, dem man in das nackte Gefieder greifen will.

„Aber es war doch ein Schatten dabei,“ fuhr
Hans Jung fort und lächelte, als er merkte, wie
alle horchten. „Eines Abends, als wir in die Wüste
geritten waren, saßen wir zusammen in meinem
Hotelzimmer und sahen nach den letzten Sonnen­
strahlen. Palmen standen vor dem Fenster, und mein
Freund hatte über den Fleiß der Abendländer ge­
lacht, und daß sie es nicht fassen könnten, daß man
sein müsse wie die Palmen, still und stolz – und
mit der gleichen ruhigen Krone unter der Sonne und
unter der Nacht. Später stand ich auf, um ihm ein
kleines Werk über die Kunst der Renaissance zu
zeigen, das gerade auf meinem Tisch lag. Ich schlug
zufällig eine Seite auf, die das Bild eines Frauen­
kopfs zeigte. Ich weiß den Namen des Meisters
nicht mehr; es war nichts als der Kopf einer jungen
[s 122]
Frau, ein wenig herb und halb von der Seite ge­
sehen. Mein Freund sah es kurz an und sagte leis:
»Das gibt es,« aber später hob er das Bild an das
letzte Licht des Abends und blickte darauf, bis es
finster geworden war und senkte die hohe Stirn
seines Abenteurerkopfes tief in den Schatten.“<l ägypten2_ende>

Er schwieg.

Die Sonne war über die stillen Hände auf der
Decke gewandert und lag an einer anderen Stelle.
Das junge Mädchen aber durchbrach das Schweigen,
sie bog den Kopf weit zurück, daß sich das Kleid
über ihrer Brust spannte und sagte mit ihrer weichen
Stimme: „Gäbe es doch Handwerksburschinnen!“
Und alle lachten mit ihr.

Der Baron begann von einer „Handwerks­
burschin“ zu erzählen, die er gekannt hatte; es war eine
Russin, die zwanzig Jahre lang ziellos in der ganzen
Welt herumgereist war mit dem Gelde ihres Mannes,
der wenige Wochen nach der Hochzeit gestorben war;
dann war sie nach Paris gezogen und hatte ein großes
Haus geführt, wo man die seltsamsten Damen und
Herren der ganzen Stadt traf; sie liebte das Groteske
vor allem und ihre Feste waren berühmt. Auch
Doktor Norten hatte sie gekannt; er erzählte von
einem Maskenfest, zu dem sie ein Hotel in Versailles
gemietet hatte und das drei Tage lang dauerte.

[s 123]
Frau Marie erhob sich, sie gingen über die
sommerhellen Kieswege ins Haus, um zu musizieren;
die drei Herren gingen hinter den Frauen, der alte
Baron in der Mitte. Und Hans Jung sah auf ein­
mal, wie Doktor Norten, der beim Gehen aussah
wie ein alter Zwerg, seine Augen aufriß und nach
Frau Maries Rücken starrte wie auf ein er­
schreckendes Bild. Aber seine Schwester hatte den
Arm der Frau genommen und sah sie bewundernd
an, und Frau Marie senkte den Kopf und horchte
lächelnd, wie das junge Ding sagte: „Sind Sie erst
als Frau so schön geworden?“

Auf der Schaukel vor der Treppe saß ihr Bub
und flog in der Sonntagssonne hin und her. Sein
Rücken streifte bis in die Blätter des Busches, der
hinter der Schaukel stand. Sie blieben ein wenig
stehen und bekamen alle den gleichen verträumten
Ausdruck in die Augen, während der Bub sich still
weiter schwang.

Berta Norten trat zu ihrem Bruder, der neben
Frau Marie stand und auf ihre Hand blickte, als
wolle er sie fassen. Sie strich ihm kurz über das
schöne Haar und sagte: „Weißt du noch die Schaukel
am Brucker Weiher, Karl?“ Er nickte und lächelte.
Und auf der Treppe sagte sie zu Frau Marie: „Wir
hatten daheim eine Schaukel, da flog man, wenn sie
[s 124]
hoch hing, vorn über das Wasser eines kleinen Sees
und hinten in die Zweige einer Weide.“ Aber auch
die Frau erwiderte ihr nichts und lächelte nur.

Doktor Norten war gern bereit, mit Hans Jung
eine Suite von Bach zu spielen, zu der sie die Noten
hatten. Aber es mußten ein paar Kissen auf den
Klavierstuhl gelegt werden, damit er hoch genug
sitzen konnte. Dann begann er. Und Hans Jung
merkte schon an dem kurzen Vorspiel zum ersten
Satz, wieviel Musik in den bleichen Händen lag, die
viel zu groß für den schmächtigen Körper des Doktors
waren. Aber dann kam sein Cello, und das Klavier
wob sich klar und ruhig in seinen Gesang.

Die Frauen und der alte Mann saßen im Erker
und horchten.

Als sie geendet hatten und in den Erker zurück­
kehrten, unterbrach Berta Norten zuerst das Schweigen,
sie hob die Hände ein wenig und sagte, während
sich ihr Bruder den Schweiß von der bleichen Stirne
wischte: „Ich danke Ihnen, Frau Marie.“

Später spielte Doktor Norten eine Chopinsche
Nocturne und blieb zusammengesunken auf dem
hohen Stuhl sitzen, als die letzten Töne schon lange
verklungen waren.

<s 0.18>Das Gespräch war versiegt, und sie saßen eine
Weile schweigend in dem sonntäglichen Zimmer, ehe
[s 125]
sie sich erhoben, um zu gehen. Hans Jung ließ sein
Cello am Flügel stehen; niemand beachtete es. Frau
Marie begleitete sie bis zum Nußbaum am Gartentor.<s 0>

„Kannten Sie die Suite schon?“[|,] fragte Hans Jung
den kleinen Doktor. Sie gingen zusammen hinter
dem Baron und dem jungen Mädchen durch die Wiesen.

„Ja,“ erwiderte er und streifte mit seinem elfen­
beingriffigen Spazierstöckchen durch die Gräser am
Weg.

„Aber Frau Doktor Blank kannte ich noch nicht,“
sagte er und sah mit einem schnellen Blick zu Hans
Jung hinauf. „Ich kannte nur ihren Mann, bei dem
ich eine Zeitlang in Behandlung war.“ Er schwieg
kurz, ehe er weiter sprach: „Später sah ich ihn dann
öfters mit dicken Mätressen zusammen. – Kennen
Sie ihn auch?“

„Ja,“ sagte Hans Jung kurz und hart, während
sie das Mädchen vor ihnen auf·lachen hörten.

„Ach, wir haben ja Bach zusammen gespielt –“[|,]
sagte der Kleine plötzlich und seine Stimme klang
ein wenig gepreßt; er stockte und hemmte seine
Schritte, daß sie hinter den anderen zurückblieben.
Hans Jung wußte nicht, auf was er hinaus wollte,
ehe er seinen Satz vollendete: „Wissen Sie, wie
solche Frauen zu solchen Männern kommen?“

Hans Jung gab ihm keine Antwort. Er sah
[s 126]
über die Wiesen hin, auf die schon vom Berg her
die Abendschatten rückten.

„Und Kinder von ihnen bekommen,“ fuhr der
Kleine eigensinnig fort. Aber es klang doch eine
tiefe Traurigkeit unter seinen Worten.

Hans Jung sagte: „Schauen Sie, wie seltsam
dort oben die einzelnen hellen Bäume unter dem
dunklen Laub stehen.“

Die Grillen begannen zu zirpen. Das Mädchen
vorn bückte sich und pflückte ein paar Kornblüten,
die zusammenstanden. Der Baron stand daneben
und sah auf ihren Nacken. Sie gab ihm den kleinen
Strauß in die Hand, und Hans Jung hörte, wie er
sagte: „Sie hat keine Freundin; es wird sie freuen,
wenn Sie manchmal –“ Hans Jung horchte nicht
länger. Er sah zu dem Doktor herab, der versonnen
vor sich hin blickte. Und auf einmal war es Hans
Jung, als erwache er aus einem Traum, der voller
Gleichgültigkeit war, in die Wahrheit zurück – und
die Brust wurde ihm schwer wie seit dem Frühling
nimmer.

Der Kleine aber hatte wieder zu sprechen be­
gonnen. Er sagte mit seiner traurigen, alten Stimme
und doch in einem Ton wie ein trotziges Kind: „Ich
habe schon manche ähnliche Mißverhältnisse gesehen
– und ich fand, daß die feinsten Frauen am schwersten
[s 127]
unter einem unwürdigen Mann leiden. Und daß
gerade die feinsten Frauen am schwersten sich wieder
frei machen können –“

Hans Jung war einige Schritte zurückgeblieben,
um eine Kornblüte aufzuheben, die der Baron fallen
gelassen hatte. „Schauen Sie,“ sagte er lachend,
„so gehen die alten Herren mit den Gaben junger
Mädchen um.“

Aber es war, als wolle der Bucklige in das junge
Gesicht hinaufschlagen, das abweisend zu ihm herab­
lächelte: „Vielleicht werden sie nimmer frei, weil sie
in dem gleichen Bett schlafen müssen.“ Und er ließ
plötzlich den Kopf sinken, als schäme er sich, es gesagt
zu haben.

Am Hotel verabschiedeten sich die Geschwister
und gingen zusammen in das leere Vestibül. Berta
Norten trat an einen Busch Chrysanthemen und stahl
eine von den Blüten, nachdem sie gesehen hatte, daß
niemand in der Nähe war. Ihr Bruder starrte auf
die Goldfische in dem kleinen Springbrunnen, der
durch die Halle plätscherte. „Das war ein schöner
Nachmittag,“ sagte sie und steckte ihm die breite
Blüte in die ausgewölbte kranke Brust. „Komm
mit,“ sagte sie und griff ihn am Arm. „Arbeite
nicht heut.“ Er folgte wie ein Kind und ließ sich
bis in die Dämmerung von der großen Schwester
[s 128]
über den klaren See rudern. Sie sang leis, wenn
Boot und Ruder trieben.

Hans Jung begleitete den Baron die Dorfstraße
zurück. Er wurde wieder zum Abendessen einge­
laden, aber er mußte danken. Als er ging, nahm er
die Hand des alten Mannes, als wolle er sie liebkosen.
Er lief fast den Wiesenweg zu ihrem Haus zurück.

Das Tor war offen. Sie stand mitten im Garten
bei einem verblühten Strauch, still wie eine Säule.

Er nahm den Hut in die Hand und versuchte
zu lächeln. „Ich habe mein Cello vergessen,“ sagte
er heiser, „so mußte ich wiederkommen.“

„Komm,“ sagte sie leis. „Wir wollen es holen.“

Sie saßen im Erker, bis sich die Dämmerung
über die Rosen schlich. Sie saßen stumm, zuweilen
nur hob sie den Kopf und sah ihn an, und er nahm
ihre Blicke auf, wie man Blüten auf sich regnen läßt
unter einem Baum.

Es war dunkel geworden, als er an die chinesische
Kette dachte. Er hüllte sie aus dem Tuch und streifte
sie langsam über ihre Hand.

„Sie gehört dir,“ sagte er leis.

Sie hob die Hand, daß die Perlen in schlankem
Bogen von den Gelenken hingen.

Er sagte: „Sie ist alt . . andere Frauen haben
sie vor dir getragen.“

[s 129]
„Welche Frauen?“[|,] fragte sie.

Er sagte: „Östliche Frauen . . auf deren Gelenke
östliche Dichter ihre Lieder sangen.“

„Waren sie glücklicher als ich . . ?“[|,] fragte sie
schwer.

Da sank er vor ihr zu Boden und beugte den
Kopf in ihren Schoß, und es klang wie vor einem
Altar: „Ich gehöre dir . .“[|,] und sie strich ihm übers
Haar, daß die Perlen leise klirrten.

Die Mägde waren zum Tanz; es war dunkel
und still. Sie gaben sich die Hände und gingen über
die Stufen. Mond lag auf den Wegen und auf dem
First des Daches. Das Gartentor fiel ins Schloß
zurück; sie gingen zum Wald. Kleine Wolkenschollen
umtrieben den Mond, daß das Licht in den Tannen
irrte. Sie standen still bei einem alten Baum, und
seine dunkle Krone bot sich wie ein hohes Dach.
Er sagte: „Lehn' still . . dein Haar ist weich . .
kein Weg führt mehr aus diesem Wald.“ – „Ich
kann nicht sehen, ob deine Augen traurig sind.“ –
Er sagte: „Laß deine Hand zu ihnen heben . . .
müssen sie nicht voll Singen sein.“
[|<aaa>]
Dichte Wolken waren vor den Mond gezogen,
als Hans Jung von ihrem Tore ging. Das Licht war
von den Wiesen gegangen. Nun lag es irgendwo
weit weg auf anderen Bergen. Seine Schritte hallten
[s 130]
wieder zwischen den stillen Häusern im Dorf. Nur
in wenigen war noch Licht. Die meisten schliefen tief.

Als er langsam die Treppe zu seinem Zimmer
hinaufgegangen war, sah er auf einmal eine Gestalt
am Geländer lehnen. Es war Marie. Sie stand
still und rührte sich nicht. „Gute Nacht,“ sagte er
leis, „sind Sie noch nicht zu Bett?“ Sie rührte sich
nicht. Aber er sah, daß sie den Kopf in den Nacken
bog und die Augen geschlossen hatte. Da zog er
sie kurz an sich. „Gute Nacht,“ flüsterte er. Er mußte
sie fast von sich drängen, so schmiegte sie sich an seine
Brust. „Sei mir nicht bös heut,“ sagte er, „gute Nacht.“

Es war ihm, als käme er von einer langen Reise
zurück, als er in sein Zimmer trat. Er hörte die
Tür im Nebenzimmer gehen, ein paar Schritte drangen
leis herüber. Dann ward es still im ganzen Haus.
Er trat ans Fenster.

Sein Schatten stand einsam in dem hellen Rahmen,
zusammengesunken.

Aber auf einmal reckte er sich . . Was nützte
es, über den Tag zu grübeln, wenn es noch tief in
der Nacht war. Jetzt war anderes zu tun. Er warf
den Kopf weit in den Nacken zurück.

Was mußte er tun? – In dem Dunkel liegen
und träumen, wie nach einem Fest, dessen Geigen
wieder tönen und dessen Kerzen wieder flackern sollen.

[s 131]
Was mußte er träumen? – Daß eine Frau von
tausend Seiten einer Halle auf ihn zugeschritten kam
und alle Säulen klangen, an denen sie vorüberging.

Was mußte sie singen? – „Du.“
[|<aa>]
Die Tage wurden heiß und schwül, die Wiesen
sengten in der Sonne. Hans Jung ging durch die
Glut der Nachmittagshitze zu Frau Marie.

In der Nähe ihres Hauses sah er plötzlich ein
großes, schlankes Mädchen vor sich gehen, die Schwester
des kleinen Doktors. Sie trug einen weitrandigen
Sommerhut und wiegte sich beim Gehen leis in
den Hüften.

Er blieb hinter einem Baum stehen und wartete,
bis sie durch das Gartentor ging. Dann wandte er
sich und ging nach der anderen Seite, den Weg zum
Wald empor.

Er wollte die beiden Frauen gern allein lassen.
Er hatte noch Zeit, bei Frau Marie zu sein, viele
Stunden. Vielleicht war es besser, am Scherzo=Satz
seines Quartetts zu arbeiten. Irgendwo in der Nähe
seines Wegs hörte er einen Bach fließen. Er bog
in die Bäume ein und kletterte über Gestrüpp und
halbverfaulte Baumstümpfe dem Rauschen nach.

Es war ein kleines, steiles Wasser, die Bäume
waren licht an seinen Ufern. Hans Jung legte sich
ins Moos, zwischen die kleinen überwachsenen Steine.

[s 132]
Wie die Schwüle zunahm rings. Selbst der
Bach sah heiß und schläfrig aus. Das Moos war
warm von der Sonne durchflutet, ein paar Käfer
summten schwer darüber hin. Die Luft war dick
wie vor einer großen Brunst.

Mochte sich ringsum alles verbinden, bis es müde
war . . er wollte allein sein und schaffen . . selbst
nicht an die Frau denken, um die die Luft klar und
köstlich war . . nichts fühlen als sich – sich.

Berta Norten sagte zu Frau Marie: „Darf ich
ein wenig bei Ihnen sitzen?“ Sie setzten sich in
den sonnigen Erker und Frau Marie lächelte still,
wenn der offene Blick des Mädchens bewundernd
an ihr hing.

„Es ist nur um Lebewohl zu sagen,“ sagte
Berta Norten.

„Wollen Sie schon abreisen?“

„Ich muß mit meinem Bruder gehen,“ sagte sie,
„er will heim.“

Frau Marie antwortete nicht. Sie nahm eine
von den Chrysanthemen, die in der Kopenhagener
Vase auf dem Tisch standen und spielte leis damit.
Sie waren von Hans Jung. Sein Cello stand hinter
dem Flügel, und der Blick des jungen Mädchens,
der über das Zimmer streifte, blieb einen Augen­
blick daran haften.

[s 133]
„Ihnen muß es immer gut gehen,“ sagte Frau Marie.

Das Mädchen senkte den Kopf und schwieg.
Und erst nach einer kleinen Weile sagte sie leis:
„Es gibt viele schöne Worte, die sind dumm ge­
worden, weil sie über tausend unwürdige Lippen
gehen. Aber sie können doch schön bleiben, wenn
man sie richtig hören kann. Kennen Sie das, Frau
Marie?“

Die Frau sah sie fragend an. „Ja,“ sagte sie
langsam.

Berta Norten hob den Kopf ein wenig und
ihre jungen Augen lächelten still. Sie sagte: „Ich
habe einen Bräutigam.“

[- s]Frau Marie sah nicht auf. Sie strich über den
Blütenschirm der Chrysanthemen, als wolle sie ihn
kosen. Und erst als sie eine Zeitlang still gesessen
waren und das Mädchen nicht weitersprach, sagte
die Frau, und es war nichts Neues zu ihrem vorigen
Ton hinzugekommen: „Ihnen muß es immer gut gehen.“

Das Mädchen erzählte von ihrem Bruder. „Ich
weiß nicht, warum er plötzlich abreisen will,“ sagte
sie. „Vielleicht kann er hier nicht arbeiten. Er
spricht ja nie darüber. Aber er sagte früher einmal,
er brauche oft den Lärm der Stadt, um schaffen zu
können und tausend Leute, die an ihm vorüber­
gehen, ohne ihn zu kennen und zu beachten.“

[s 134]
„Was arbeitet er?“[|,] fragte Frau Marie.

Man konnte nicht hören, ob es unbewußt geschah,
daß das Mädchen seine Worte so seltsam verband:
„Er ist Dichter; und er läßt Sie grüßen. Er konnte
leider keine Zeit finden, selbst zu kommen, um sich
zu verabschieden. Aber ich soll Ihnen dies mitbringen.“

Sie nahm ein schmales Büchlein von dem Stuhl,
auf den sie es gelegt hatte. Es waren Gedichte.
Hinter die letzten Druckseiten waren noch einige
handgeschriebene Blätter geheftet.

„Ich danke ihm,“ sagte Frau Marie und strich
leise über den grünen Deckel des Buches. „Wollen
Sie nicht einige vorlesen?“

Das Mädchen weigerte sich nicht, sie suchte ein
wenig zwischen den Seiten und las dann aus den
Gedichten, die im Manuskript waren; sie waren frisch
geschrieben, die Tinte war noch nicht vergilbt.

Ein paar Lieder von einsamen Kindern, die
beteten; von einer Frau, die auf·lachte, während ein
Mann zu ihren Füßen lag.

Frau Marie mußte an Hans Jungs Worte über
den kleinen Dichter denken: „Man möchte gerne
mit ihm weinen – aber man tut es nicht, wenn
sie ihre Leiden benutzen, um zu erpressen. .“ Wie
hart hatte er es sagen können. . Sie hatte daran
gedacht, daß er schon auf allen Meeren gefahren war. .

[s 135]
Berta Norten blätterte wieder um.

Das Blut stieg ihr leise in die Wangen, als sie
las, ehrfürchtig und ein wenig verschämt:

<v begin 4.0cm>
Die Courtisane singt:
<v end>

<v begin 6cm>
Ich hab' ein Bett von weißer Seide,\\*
voller Lüste jede Nacht.\\
Heute kommt mein treuster Buhle,\\*
bringt mir Nelken mit und lacht.

Morgen ist's ein alter Herr,\\*
noch immer ohne Ruh.\\
Er schenkt mir einen Papagei,\\*
der sagt: „Wie schön bist du.“

Aber gestern war's ein Bursch,\\*
tausend Küsse wert.\\
Doch er kam nur zu vergessen\\*
eine Frau, die ihn nicht hört.

O, wie haß ich die Frau, die einsam schläft,\\*
während sich der Bursch verzehrt.
<v end>

Das Mädchen schloß das Buch schnell.

„Vielen Dank,“ sagte Frau Marie und gab ihr
die Hand. „Und viele Grüße. Sagen Sie ihm,
nun habe ich mich umsonst darauf gefreut, daß er
öfters käme“, und sie hob die Hand, von deren Gelenk
die Perlen hingen, bis in den breiten Streifen der
Sonne empor, „und Herrn Jung zum Cello begleite“.

[s 136]
Sie gingen in den Garten.

„Wie schwül es ist,“ sagte das Mädchen.

„Aber der Abend wird kühl und klar,“ sagte
Frau Marie und atmete auf, daß sich ihre Brüste hoben.

Auf der Decke in der Laube lagen schon einige
gefallene Blätter.

„Sehen Sie,“ sagte die Frau, „nun wird der
Sommer bald vorüber sein.“ Und sie lächelte leis,
„wie alt ich schon geworden bin“. Aber es war,
als wolle sie das Gesagte schnell verwischen, als sie
weiter sprach: „Heute ist es selbst den Amseln zu
warm, um zu singen. Die Schwüle macht alles so still.“

Das junge Mädchen hatte nur ihre ersten Worte
gehört, sie beugte den Kopf mit dem glatten, braunen
Scheitel tief auf die Decke herab und küßte die Hand
der Frau.

„Schreiben Sie einmal,“ sagte Frau Marie, als
sie ohne Hut mit durch die Wiese gegangen war
und sich wandte, zurückzugehen.

„Danke, gern,“ sagte Berta Norten, und sie
errötete plötzlich bis unter den Rand des Sommer­
hutes und sagte: „Aber ich werde nichts zu schreiben
wissen, als Nebensächlichkeiten –“

„Und“[|,] sagte Frau Marie und ließ ihre Hand
nicht los.

„Und daß ich warten muß.“

[s 137]
Die Frau sagte: „Sie wissen, worauf Sie warten
müs­sen[,|] <>–<> an­dere gibt es, die warten, und wissen
nicht, auf was.“<s 0>

Aber sie schritt durch die Wiesen zurück, als
tanze sie einen leisen Tanz[. . . .
|.
<a>]
<s 0.14>„Was denkst du, wenn du so sitzt?“[|,] fragte
Hans Jung. Er kam am Abend mit einem Pack Noten
und einem frischen Waldblumenstrauß und traf sie
allein in der Laube, die Hände sinnend im Schoß.<s 0>

„Was ich denke?“[|,] sagte Frau Marie und strich
über die blauen und roten Blumenköpfe.

Er stellte sich vor sie, daß sie vom Garten aus
nicht mehr gesehen werden konnte. „Was du denkst?“[|,]
fragte er leis.

„An das, was ist,“ sagte sie, ohne den Kopf
zu heben.

„Und?“

„Und an das, was nicht ist.“

„An was?“

Sie blieb still.

„An ein Floß,“ sagte sie dann, „das auf einem
Strom gleitet, unter den Zweigen der Uferbäume hin.“

„Und?“

„Und daß die Äste schwer voll Blüten über
das blaue Wasser hängen.“

„Und?“

[s 138]
„Daß das Floß uns gehört.“

„Und?“

„Ich weiß nicht mehr . .“

„Und daß du singst über den Fluß,“ sagte er.

Sie schwieg.

„Und daß die Leute am Ufer stehen bleiben,
wenn sie dich sehen und rote, blasse Blüten brechen,
um sie auf dein Floß zu werfen.“

Frau Marie senkte den Kopf tief.

„Aber wir lachen ihnen nur zu und gleiten weiter.“

Er strich ihr über das Haar und hob ihr den Kopf.
[
|<aa>
]<s 0.22>[. . |]Wenn es dunkel geworden war, saßen sie still
im Erker und schwiegen. Sie hatte Schubert=Lieder
gesungen, und es war, als ob ihre Stimme voller
und fraulicher würde in den letzten Sommertagen.
Dann war er mit ihr gewesen, wenn sie ihren Buben
entkleidete und zu Bett brachte. Und später hatte
er Cello gespielt; es war immer das gleiche, was sie
in den letzten Tagen hören wollte: eine kleine
Cavatine, die er für sie geschrieben hatte. – „Ich
verstehe sie nicht,“ sagte sie, „und doch höre ich
ganz, wie schön sie ist.“ – Er lächelte. „Du irrst,
sie ist nicht viel wert,“ und er nahm sie schnell
bei der Hand. „Aber einmal . . später . . möchte
ich etwas für dich schreiben, das mich selbst durch­
schauert, so oft ich es höre. . .“<s 0>

[s 139]
Einmal spielte er ein kleines Schlummerlied, das
Frau Boucher gewidmet war.

„Wer war das?“[|,] fragte sie.

„Das war in Paris,“ sagte er, „eine Frau, in
deren Haus ich öfters kam.“ Er lachte plötzlich auf.
„Wie dumm ich war, damals komponierte ich nur
Schlummerlieder.“

Aber Frau Marie lachte nicht mit ihm. Da
sagte er ernst: „Es war eine kranke, junge Frau.
Im Sommer fuhr sie ans Meer, dann konnten für
kurze Wochen blasse Rosen auf ihren Wangen
blühen. Aber im Winter litt sie sehr und saß am
Kamin, wenn ich spielte, denn sie fror immerzu. Ich
mußte immer wieder die slavische Elegie spielen, die
sie vor allem liebte. Und manchmal wob sich in die
schwere Melodie wirr der Husten, der ihre magere
Brust schüttelte.“

Er brach ab.

Frau Marie fragte leis, als rühre sie an ein
kostbares Gefäß: „Wie geht es ihr?“

„Sie ist schon lang gestorben,“ sagte er.
[
|<a>
]<s>[. . |]Baron Mannen kam nicht mehr jeden Tag
zu Frau Marie. Er mußte viel arbeiten, sagte er.
Aber wenn er kam, war er so liebenswürdig, daß
sich Hans Jung selbst öfter dabei ertappte, wie seine
Stimme klang, als wolle er die alte Hand streicheln.

[s 140]
„Ich will eine Reise machen im Herbst,“ sagte
der Baron. Sie saßen in der Laube beim Kaffee.

„Aber ich weiß noch nicht wohin,“ fuhr er
lächelnd fort. „Wohin raten Sie mir, Frau Marie?“

Sie besann sich ein wenig.

<l fremdenzimmer>„In unser Fremdenzimmer in der Stadt,“ sagte
sie. „Jeden Tag wird ein frischer Strauß Herbstzeit­
losen auf dem Tisch stehen.“

Er dankte lachend und warf ihr eine kleine
Knospe, mit der er gespielt hatte, an die Brust –
wie ein junger Bursch.

„Und wohin raten Sie, Herr Jung?“[|,] fragte er.

„Ich rate Ihnen zu den Fremdenzimmern, die
ich vielleicht bis dorthin in einem Haus in Versailles
oder bei den Tuilerien eingerichtet habe.“

„Und was wird dort auf dem Tisch stehen?“

Hans Jung lachte: „Seinerosen, Tabak und
Tintenzeug.“

„Küß die Hand,“ sagte der Baron; aber das
lustige Auf·lachen paßte nicht zu seinen Worten: „Ein
Fremdenzimmer irgendwo wird es schon werden.“

Sie begleiteten ihn bis an das Gartentor.
„Kommen Sie doch einmal zusammen zu mir,“
sagte er. „Am Nachmittag oder am Abend.“[ –|]

Sie gingen hinauf, um eines von Hans Jungs
Liedern auszusuchen, das sie dem alten Mann bringen
[s 141]
wollten. Sie fanden ein Lied mit einer kleinen,
schlanken Melodie, das von nichts handelte als von
Pflaumenblüten, die in einen tiefen Brunnen fielen.
[
|<aa>
][. . . |]„Wollen wir ein wenig rudern?“[|,] fragte
Hans Jung Marie. Sie standen auf dem Rasen
hinter dem Haus. Es war, als sähe er mit seinen
eigenen Augen die wehmütigen Schleier fortfliegen,
die ihr stummes Wesen in den letzten Tagen um­
geben hatten. Die Morgensonne glänzte durch ihr
goldenes Haar.

„Ich komme gleich,“ sagte sie und lief ins Haus.
Sie hatte das rotseidene Tuch um die Schultern,
als sie wiederkam.

Die Ruder ächzten. Sie fuhren langsam dem
Ufer entlang, unter den Zweigen der Bäume. Marie
saß an der Steuerbank und sah manchmal mit einem
stillen Lächeln zu ihm auf. Er sah auf einmal, daß
sie viel geweint hatte in der letzten Zeit.

Er ließ die Ruder gleiten. „Trägst du die
indische Kette nie?“[|,] fragte er. – „Ich trage sie oft,“
sagte sie und lächelte froh. – „Du mußt zu mir
kommen, wenn wir wieder zu Hause sind. Ich habe
noch eine andere, die dir vielleicht besser gefällt.“

Ein anderes Morgenboot kam vorbei, zwei
junge Fräulein saßen darin. Sie hatten die gleichen
blauen Augen und das Haar auf die gleiche Art
[s 142]
in der Mitte gescheitelt und nackte braune Arme.
Und sie hatten grünes Laubwerk im Boot und
sangen zusammen. „Grüß Gott,“ sagte Hans Jung
und fuhr so dicht an ihnen vorbei, daß sie die
Ruder einziehen mußten. „Grüß Gott,“ sagten die
Schwestern und sahen mit ihren unverhüllten Augen
in das fremde Boot. „Grüß Gott,“ sagte Marie leis.

Es war eine kleine russische Münzenkette, die
er ihr schenkte, als sie zurückkamen. „Marie, Marie,“
rief die Mutter auf der Treppe. Nun mußte sie gleich
wieder hinuntergehen, und es war ihm bequem so.

<l waldspazierfahrt_anfang><s>Er arbeitete an dem Adagio=Satz seines
Quartetts. Er saß und hörte nur die Geigen klingen
und hatte alles andere vergessen. Er hatte vergessen,
daß bald ein Wagen kam mit den zwei besten Pferden,
die er im Dorf hatte finden können; sie wollten
weit in das tiefe Tal hineinfahren, das sich zwischen
die hohen Berge grub. Aber er dachte nicht mehr
daran, er saß in dem Duft der Wiesenblumen ver­
sunken und hörte nur das Adagio noch . . Er lehnte
sich zurück. Vielleicht würden es die Leute einmal
Spätsommer=Adagio nennen. . . Er lächelte über
sich selbst.

Marie führte die fremde junge Frau die Treppe
hinauf. An der Tür des Herrn knickste sie und
wurde rot bis unter das Haar, als sie sich wandte.

[s 143]
„Danke, daß du gekommen bist,“ sagte Hans
Jung. Die Frau war ans Fenster getreten und sah
auf den See. Er stand hinter ihr.

„Ich störte dich,“ sagte sie.

Er lachte. „[J|I]ch schrieb nur dummes Zeug.“

Er nahm ein breites chinesisches Seidenband
aus der Kommode und schlang es ihr um den Hals,
daß die Enden auf die Brust hingen. Sie stand still
und bot sich seinen leisen Händen. Silberne Fasanen
waren in die Seide gestickt, die aus schlanken, blaß­
blauen Büschen schrien.

„Du wirst wenig lesen können,“ sagte er lachend,
als sie sich über die frisch geschriebenen Noten
beugte. Die Blätter sahen aus wie Ornament·tafeln,
über die die Spuren seltsamer Vögel gehn. Sie stand
still und hielt den Kopf gesenkt, und seine Brust
wurde ihm schwer vor der welkenden Schwermut,
die um ihren stillen Körper war. Da brachte er die
Mappe mit seinen fertigen Kompositionen und ver­
streute die Notenblätter über den Tisch, alles was
er seit jenen Schlummerliedern für Frau Boucher
geschaffen hatte; Lieder, Sonaten, Quartette. Und
er nahm jedes einzelne Blatt, die frühen, vorfrühlings­
haften und die letzten, bis zur Cavatine und schrieb
mit der gleichen Schrift in alle Ecken: Für Frau Marie.

Sie lächelte nur. „Damals kanntest du mich ja
[s 144]
noch nicht,“ sagte sie und sah auf eine kleine
Romanze. – „Aber ich schrieb es doch für dich,“
und er schrieb lachend ihren Namen.

„Kennst du diese Sonate?“[|,] fragte er. Es war
eine Sonate für Geige und Klavier. Er schlug den
Satz mit den Variationen auf und summte ihr das
Haupt·thema vor. „Das war eine Zeitlang mein
Liebstes,“ sagte er. „Jetzt habe ich schon lang nimmer
daran gedacht.“ Sie hatte die Hand auf den Tisch
gestützt und horchte. Er schrieb: Für Frau Marie.
[|<aaa>]
[. . |]Der Kutscher war jung und stark und lachte
über den ganzen Mund, wenn Hans Jung mit ihm
sprach. Als sie in den Wald gekommen waren, hörte
er auf mit der langen Peitsche hin und her zu knallen;
er steckte sie beiseite und nickte ein wenig ein. Die
Sonnenflecken glitten leis über seinen grünen Hut.

Die Pferde kannten den Weg. Er stieg wenig
[uud|und] ging immer geradeaus. Sie durften so langsam
gehn, als sie wollten und nickten mit den Köpfen.
Die Kornblüten, die Hans Jung hinter die Ohren
der Stute gesteckt hatte, waren schon längst ver­
loren.

„Hast du eine Schwester?“[|,] fragte die Frau auf
einmal leis.

„Nein.“

Er lächelte ein wenig. „Ich will niemand . .
[s 145]
nicht einmal einen Weg mehr will ich . . alles ist
zu einem geworden . . Tag und Nacht und Süden
und Norden . . es ist zum Rahmen geworden –
um dich.“

Ein Käfer summte eine Zeitlang um den Wagen.
Dann flog er fort, und sein Summen verlor sich
hinter den gesenkten Köpfen.

Die Frau sagte:

„Meine Mutter war eine seltsame Frau . . sie
ging still durch das Haus, und doch sah man, daß
sie zum Singen auf die Welt gekommen war.

Mir schien sie schöner als alle anderen Frauen
. . aber dann mußte ich fort von ihren leisen Händen,
die über unsere Haare strichen, wenn sie erzählte.

Mein Mann war ein Vetter von ihr . . ich sah
ihn erst, als er schon berühmt war, weil er so gut
operierte. . . Als Mädchen stellte ich mir eine
Operation wie einen herrlichen Kampf vor.“

Der Kutscher war ein wenig wach geworden.
Er nahm die Peitsche und kitzelte den Pferden ein
paarmal um die Ohren. Aber dann träumte er weiter.

„Schau dort, der Baum blüht noch,“ sagte
Hans Jung.

Sie wandte den Kopf.

Erst als es Nacht geworden war, wendete der
Kutscher den Wagen. Die Pferde spitzten die Ohren
[s 146]
und drehten die Deichsel Schritt für Schritt. Die
Räder ächzten. Die Laterne warf einen trüben Schein
auf die Bäume und Sträucher am Weg. Hans Jung
nahm die Hand der Frau. Und sie hörten nicht
mehr das Traben der Pferde, die nach dem Stall
liefen und nicht die Büsche, die die Wagenseiten
streiften und nicht den Burschen auf dem Bock, der
wach geworden war und wieder mit der Peitsche
durch den Wald knallte.

<l knallen>Ganze Lieder konnte er knallen.<l waldspazierfahrt_ende>
[
– –
|<aa>]
Wenn die Vögel im Garten wach wurden,
mußte er gehn. Nun schliefen sie noch. Und die
Grillen in den Wiesen hatten aufgehört durch die
Nacht zu lärmen. Die Fenster des großen Zimmers
waren offen, und der Wind kam aus dem Garten
und hob still die hellen Gardinen. Hans Jung wachte.
Der Kopf der schlafenden Frau lehnte an seiner
Brust, und er wagte nicht aufzuatmen unter der kost­
baren Last. Seine Augen waren traurig, daß sie ihn
brannten. . . Es war, weil er daran dachte, daß sie
zurückbleiben mußte und alt werden . . welken wie
die Blüten, die in den tiefen Brunnen gefallen waren.
Er senkte sein Gesicht still in ihr Haar. Es war
ein Duft darin, leis und herb.

Der erste Schein der Dämmerung schlich durch
[s 147]
die Gardinen. Da hob die Frau den Kopf und
sah ihn an.
[
|<aa>
][. . . |]Das Dorf schlief noch; nur die Hähne
begannen zu krähen. Und der Himmel hatte seine
erste Bleiche verloren. Hans Jung ging langsam und
gebeugt zwischen den Häusern; es war als trüge er
eine Last aus der Nacht. Aber auf dem stillen Weg
am See verlor sie sich von seinen Schultern. Er
atmete tief den köstlichen Morgenduft, der über den
Wassern und zwischen den Bäumen hing, und seine
Finger streiften spielend in den hohen Gräsern am
Weg. Der Himmel wurde dunkelblau, die Sonne
kam aus den Bergen. Weit draußen war eine kleine
Wiese, die flach zum Wasser fiel und von einem
Halbkreis schlanker Bäume umringt war. Er ent­
kleidete sich mitten in dem sonnigen Tempel. Ein
paar Käfer umsummten ihn lustig dabei, alle Vögel
sangen im Wald, irgendwo schrien Wildenten. Hoi=ho
– das Wasser war kalt und hart, es wollte ihn
zurückweisen. Aber sein Körper war straff, er stürmte
hinein und schwamm weit in den See, bis sein Kopf
vom Tempel aus nur noch wie ein schwarzer Punkt
im Glast der Sonne schien. Das Wasser umspielte
ihn wie ein silberner Freund. Und es war hell und
klar, daß man weit in seine grüne Tiefe hinunter­
sehen konnte.

[s 148]
Er jauchzte auf, er fühlte nichts als sich, seinen
Körper und sich – sich.
[
– –|<aa>
] Hans Jung stand in der Stube bei der
Mutter Maries. Sie saß am Fenster und sah auf die
tiefen Wolken hinaus, und ihre bleichen Hände lagen
still in den Falten des Schoßes. „Nun ist das schöne
Wetter wieder vorbei,“ sagte sie. „Setzen Sie sich
nicht ein wenig, Herr Jung?“

Er setzte sich auf den Stuhl ihr gegenüber.
„Nun werde ich bald fortgehen von hier,“ sagte er.

„Wohin?“[|,] fragte die alte Frau und sah ihn an,
als wolle sie aus seinen Mienen lesen.

Hans Jung besann sich einen Augenblick.

„Nach Paris,“ sagte er dann.

„Paris,“ sagte sie und horchte noch einmal, wie
fremd das klang.

„Ich komme aber wieder zurück,“ sagte er, „bis
dorthin wird hier vieles anders sein.“

„Was?“

Er lachte ein wenig. „Fräulein Marie wird dann
verheiratet sein,“ er mußte sich besinnen, ehe er weiter
sprach; seit den letzten Tagen war es ihm ein wenig
mühsam, Gespräche zu leiten. „Vielleicht komme
ich gerade zur Hochzeit.“

„Das glaube ich nicht,“ erwiderte sie. „Marie
ist eine Besondere. <l kein_bursch>Ihr paßt kein Bursch.“

[s 149]
Hans Jung stand auf.

„Sie hat recht, daß sie stolz ist,“ sagte er. „Ich
komme schon noch einmal herein, ehe ich reise.“
[|<aa>]
Er mußte zu Baron Mannen, den er eingeladen
hatte, mit ihm in das Forsthaus zwischen den Bergen
zu fahren und in der Waldschenke zu Abend zu
essen. „Das ist ein lustiger Vorschlag,“ hatte der
alte Mann gesagt.

Nun saßen sie an dem kleinen Fremdentisch,
über den die Wirtin schnell eine weiße Decke ge­
breitet hatte. Ihr Kutscher hatte die Pferde aus­
gespannt und setzte sich an den langen Bauerntisch,
unter die Forstgehilfen und Holzknechte. Sie hatten
schon Feierabend gemacht.

„Es ist zugleich mein Abschied heute, Herr
Baron,“ sagte Hans Jung, als sie eine Weile schweigend
gesessen hatten.

<s 0.20>Der alte Mann hob den Kopf ein wenig.
„Abschied?“[|,] sagte er.<s 0>

„Ja,“ sagte Hans Jung, „ich reise fort.“

„Wohin?“

„Nach Paris,“ sagte Hans Jung.

„Ach ja, in das Haus mit den schönen Fremden­
zimmern,“ sagte der Baron und lächelte ein wenig.

Es fiel Hans Jung auf einmal ein, daß es vielleicht
ganz gut wäre, wirklich eine Zeitlang nach Paris zu gehen.

[s 150]
„Dann werden wir uns wohl nimmer sehen,“
sagte der alte Mann.

„Warum?“[|,] sagte Hans Jung, „die Welt ist klein.“

Der Baron lächelte wieder. Er sagte: „Man
muß so jung sein wie Sie, um das sagen zu können.“

Am Nebentisch wurde es laut. Es war ein
alter Holzknecht da, den sie alle hänselten. Er trug
einen wirren grauen Bart und hatte hundert Falten
um seine müden Augen. Peter hieß er.

Der alte Mann sagte: „Da wir uns doch nimmer
sehen, kann ich Ihnen zum Abschied eine kleine
Geschichte erzählen. Es ist von einem Freund, der
mir gerade in den Sinn kommt.“

„Von einem Freund?“[|,] sagte Hans Jung.

<s 0.19>„Ja, von einem Freund . . einem seltsamen
Menschen . . Deswegen seltsam, weil er sein Leben
lang keine Frau anrührte . . .<s 0>

<s 0.19>Als ich ihn kennen lernte, war er mit einem
jungen Mädchen verlobt. Ich glaube, es gab kein
frühlingshafteres Bild, als wenn die beiden ihre Wege
gingen. Aber auf einmal, kurz vor der Hochzeit,
ging er von ihr. Er ging wie ein Schuft und niemand
wußte warum. Es war vielleicht, weil er einmal
ein großes Werk schaffen wollte, es sollte keine
Dichtung und keine Philosophie werden, aber etwas,
in dem beides enthalten ist; etwas Gewaltiges sollte
[s 151]
es werden. Aber dann hörte er eines Tages, daß
das Mädchen gestorben war. Sie hatte sich nicht
getötet, sie war nur jung gestorben. Er hatte schon
nimmer viel an sie gedacht. – – Aber von dem
Tage an ging er seltsam umher. Er konnte keine
Frau mehr lieb haben . . und ich glaube, er wird
sein Werk nie fertig schreiben.“<s 0>

Er brach ab.

„Ist es Ihnen kalt?“[|,] fragte er.

Es hatte Hans Jung ein wenig durchschauert.

„Nein,“ erwiderte Hans Jung.

Der Baron bat ihn, die Pferde einspannen zu
lassen.

Am Bauerntisch hatten sie die Lampe angezündet.
Die Mägde kamen und setzten sich zu den Burschen.
Sie hatten dicke, farbige Blusen und warme Köpfe.
Die Burschen lachten, daß die niedrige Stube dröhnte.
Der alte Peter ließ ihre Neckereien gutmütig über
sich ergehen. – „Wann ist die Hochzeit, Peter?“[|,]
fragten sie. – „Gebt Ruh,“ sagte er und lächelte
trüb. – „Ist die Frau schön?“[|,] fragten sie. Aber
ein ganz Junger hatte die Zither geholt und sang:

<v begin 5.5cm>
<l schnadahüpfel>Der Peter kommt geritten\\*
auf einer alten Kuh,\\
der Schwanz ist abgeschnitten,\\*
das Loch geht auf und zu.
<v end>

[s 152]
Sie bogen sich vor Lachen, daß die Mädchen
ihre roten Köpfe an die Schultern ihrer Burschen
legen mußten.

Aber der Junge ließ ab von dem Alten und sang:

<v begin 5.5cm>
Ich war schon bei der Anna\\*
und schon bei der Susanna,\\
bei Marie war ich zunachst,\\*
Teufel, ist das Luder g'haxt.
<v end>

Und als er es wiederholte, sang er ganz
melancholisch, flötete wie ein Hahn, der in dem
dämmernden Maibaum balzt:

<v begin 5.5cm>
bei Marie war ich zunachst,\\*
Teufel, ist das Luder g'haxt.
<v end>[

– –

|
<aa>
<a>
]Und dann reiste Hans Jung.

Er hatte für das erste Stück seines Weges den
Wagen mit dem jungen Bursch gemietet. Sie fuhren
ins flache Land hinaus. Der Sommer starb.

Am Abend, bevor er gegangen war, war er mit
der Frau im Erker gesessen, und sie hatten von
dem Haus gesprochen, das er für sie bauen wollte.
Einen großen Altan sollte es haben, von dem aus
man den Wald sah, und sie wollten sitzen und
warten, bis er am Abend aus seinem Zimmer kam
und ihr übers Haar strich und ihr zeigte, was er am
Tag geschrieben hatte. Und einmal sollte er ihr
[s 153]
das Werk bringen, das ihn selbst durchschauerte, so
oft er es hörte[ . . .

|.<s 0>
\\
\\
\\
\\
]Manchmal schlief der Kutscher, und die Straße
war einsam, und Hans Jung stand auf und hob die
Arme hoch und fuhr stehend durch die hellen Wiesen.
[
. . |
<aaa>
]Er konnte ihr noch lange Zeit schreiben und
danken und Lieder senden und fremdländischen
Schmuck . . . aber vielleicht auch würde die Frau
still sitzen und alt werden und nie mehr von ihm
hören.[ – –

|
<aa>
]<s 0.18>Am Abend kamen dichte Nebel. Aus den
Gehöften, an denen er vorüberfuhr, drang gelber
Lampenschein; sie saßen im Heimgarten zusammen;
schlanke Frauen und Männer mit schweren Händen.
Auch die trübe Laterne am Wagen brannte. Der
Kutscher wurde ein wenig wach und sang leis vor
sich hin ein kleines Lied.
<a>
<z begin>
Geschrieben Kairo 1911.
<z end>

[s 154]

