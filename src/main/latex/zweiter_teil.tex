<k Zweiter Teil.>


<l persien_anfang>Hans Jung saß bei Frau Marie und erzählte.
„Als ich nach vielen Monaten wieder kam,“
sagte er, „kannte mich noch das ganze Haus. Ich
ritt von Schiras aus, mein Pferd hieß Owgu und war
hellbraun wie ein Reh und streckte den Rücken, wenn
es galoppierte, als sei es ihm eine Wollust, mich zu
tragen. Als es Abend wurde, kamen wir durch einen
lichten Wald, und ich wußte, daß wir bald bei dem
Haus meines Freundes sein müßten.

Er saß vor dem Haus und sah vor sich hin und
rauchte. Es war, als sei ich gestern erst gegangen.

»Es gibt einen Gott, da du zurückgekommen
bist,« sagte er, als wir uns küßten. Nun waren seine
Haare ganz weiß geworden und seine Augen lagen
noch tiefer in ihren Höhlen.

Er fragte nicht, wie es mir gehe und wie es in
den fremden Strichen gewesen war. Er wußte, daß
man das nicht im schnellen Atem erzählen kann. Er
[s 67]
ließ Owgu von einem Knecht besorgen, der mir nicht
ins Gesicht sehen durfte. Wir gingen ins Haus.

»In deinem Zimmer hat niemand geschlafen,
seit du gegangen bist,« sagte er und führte mich in
das obere Stockwerk seines Hauses. Man konnte von
meinem Lager aus den Wald sehen, durch den ich
gekommen war. Die Sonne ging unter darin.

Mawrogeza, seine älteste Tochter, kam. Auch
ihre Haare waren grau geworden. Sie hatte mich
schon gesehen, denn sie brachte einen Korb Rosen
mit und verstreute sie über mein Lager. »Du bist
gut, weil du wiedergekommen bist,« sagte sie und
gab mir beide Hände.

Die Tage waren hell und lang. Sie vergingen
mit Reiten und Schwimmen und mit den Tänzen und
Gesängen der Frauen. Aber am Abend saß ich
allein bei dem Alten in seiner weitgedehnten Halle.
Die Frauen mußten gehen, wenn die Dämmerung
kam. Sie küßten seine Hand und gingen in ihre
Gemächer hinauf. – Mawrogeza und ihre Töchter,
Kewjy, die älteste, die einen Freier hatte, und Tero,
die jüngste, die noch keine Mädchenbrüste hatte, und
Olpje, die schönste . .

Wir saßen still und schwiegen viel und rauchten.

»Hast du viel gesehen?«[|,] fragte der Alte. »Ja,«
sagte ich. Er lächelte.

[s 68]
»Und ich habe über vieles nachgedacht,« sagte er.

Er blieb ein wenig still.

»Viel und nichts,« sagte er. »Viel, weil wenige
Menschen dazu kommen, in ihrem Leben es zu denken.
Nichts, weil es verwehen wird, wie Blütenstaub.«

Eines Abends fragte er: »Wohin willst du von
uns aus gehen?«

Ich sagte: »Ich fahre heim von hier aus.«

Er hob eine Pontroblüte aus dem Holzkästchen,
das neben ihm lag und roch daran, nachdem er sie
zwischen seinen braunen, schlanken Fingern zer­
rieben hatte.

»Sagtest du nicht einst,« er sprach langsam und
tief, »es gäbe keine Heimat für dich.«

Ich schwieg.

»Warum bleibst du nicht für immer bei uns?«

»Weil wir uns stören würden, wenn wir immer
zusammen wären,« sagte ich leise. »Du hast es
selbst gesagt.«

<s 0.18>[. . |]»Wir stören uns selbst,« hatte er früher oft
gesagt, »wenn wir nicht einsam bleiben können . .
wenn wir nach Ruhm streben . . wenn wir anderen
Menschen mehr sein wollen, als nur ein Gast.«<s 0>

»Du hast ein gutes Gedächtnis,« sagte er und
lächelte.

Wir saßen still und rochen an den Pontroblüten.

[s 69]
»Du liebst eine Frau in deinem Land,« sagte
er auf einmal.

Ich lachte.“

Hans Jung schwieg. Frau Marie rührte sich
nicht. Sie hatte den Kopf geneigt und sah aus wie
ein Mädchen, das nach etwas Verklungenem horcht.
Auf dem Gang trabte es, die Tür klinkte, ihr Bub
kam herein. Er wollte, daß sie mit in den Garten
gingen. – „Geh zu Anna,“ sagte sie und führte
ihn zur Tür zurück und schob ihn hinaus. „Anna soll
mit dir spielen.“ Sie kam langsam wieder in den
Erker zurück. – „Erzählen Sie weiter,“ sagte sie leis.

„Und an den Nachmittagen saß ich oft allein
bei den Frauen und Mädchen in der hellen Halle.
Olpje tanzte und Mawrogeza blies dazu auf der
tiefen persischen Doppelflöte.

Mawrogeza blies nur drei Töne, immerzu, auf
und ab.

Und Olpje tanzte still und andächtig wie vor
einem Altar. Ihr schlanker Körper kannte viele
Haltungen, aber sie endete ihre Tänze immer auf
die gleiche Art. Sie wirbelte im Kreis, wirbelte und
wirbelte, bis ihre kleinen Schultern zusammensanken.
Dann lief sie und verbarg sich hinter dem Rücken
der Schwestern, die lachten. Sie schämte sich, daß
sie ihre Hüften und Beine im Tanz gezeigt hatte.

[s 70]
Aber sie tanzte immer wieder.

Manchmal mußte Tero, die jüngste, tanzen.
Und alle lachten über ihre eckigen Bewegungen. Selbst
Mawrogezas faltiges Gesicht verzog sich zu einem
kleinen Lächeln, während sie ihre drei Nachtigallen­
töne weiter blies.

Aber Kewjy tanzte nie, da ihr Freier nicht
zugegen war. . .

Ich mußte von den Frauen in meinem Land erzählen.

»Die meisten Frauen in meinem Lande können
einem nicht richtig in die Augen sehn und können
nicht singen. Und wenn ein Mädchen dort tanzen
kann wie Olpje, darf sie vor tausend Menschen als
Tänzerin auftreten; so selten ist sie.«

Mawrogeza lachte und Olpje hob verwundert
die Hand und griff an ihre kleine Brust. Und Tero
starrte mich mit großen Augen an. Aber ich merkte,
daß Kewjy nicht zugehört hatte, da sie an ihren
Liebsten dachte.

<l frauenbilder>»Sie schnüren sich die Hüften ein, und in ihren
dicken Kleidern sind sie schöner als nackt. Aber sie
sind sehr stolz und nennen die fremden Frauen im
Osten und Süden Wilde. Und sie sähen aus wie
Mägde neben Göttinnen, wenn sie neben jenen fremden
Frauen zum Brunnen gehen müßten.«

Mawrogeza verzog den Mund ein wenig und
[s 71]
sagte: »Er lügt, um uns zu schmeicheln.« Aber Tero
starrte mir unverwandt auf den Mund und Olpje
hatte die Hand noch nicht von der Brust genommen.

»Es ist wahr, was ich sage,« fuhr ich fort,
»ihre Männer wollen es nicht anders.«

Olpje fragte: »Sind alle Frauen in deinem Lande
so?«

»Nein,« sagte ich, »manchmal kommt eine Frau,
die schöner ist als alle Frauen im Osten und Süden.
Aber nur wenn man Glück hat, begegnet man ihr.
Und dann bleibt alles andere nur noch Staub auf der
Straße, die sie geht.«

Ich schwieg und roch an den Pontroblüten, die
mir Tero in die Hände rieb. Olpje hatte langsam
ihre Hand gesenkt.

Aber Mawrogeza sagte auf einmal: »Du liebst
eine Frau in deinem Land.«

Ich lachte.“<l persien_ende>

<s>Es dämmerte stark. – „Soll ich Licht machen?“[|,]
fragte Frau Marie. – „Nein,“ sagte er, „wenn Sie
nicht wollen.“ Es war still im Hause. Auch von
den Mägden war nichts zu hören. Sie mochten wohl
bei ihren Burschen sein. Ein Blatt löste sich von
den Rosen, die auf dem Tisch standen und fiel
schwebend auf die braune Decke. – „Und dann
sind Sie doch gegangen,“ sagte Frau Marie.

[s 72]
„Ja“, sagte Hans Jung, „wir saßen in der Halle,
der Alte und ich, an dem Abend vor dem Tag, an
dem ich reisen mußte. Ein unruhiges Licht flackerte
aus den beiden Pfannen, die neben den kleinen Säulen
an der Tür standen. Wir tranken gekühlten Wein
aus flachen Nußbaumschalen; denn es war eine
schwüle Nacht.

Der Alte hob seinen Märchenkopf ein wenig.

»Nun haben wir viel zusammen gesprochen,«
sagte er. »Wir haben viel zusammen gesprochen.
Und ich habe viel für mich allein gedacht. Und auch
du wirst noch viel denken . . du bist ja noch jung.«

Ich saß still und schwieg. Und ich fühlte, daß
meine Stimme jetzt nicht durch den Saal klingen durfte.

»Aber unser Denken ist schwach,« fuhr er fort,
»unser Körper ist schwach. – Auch unsere Künste
sind schwach.« Seine Worte verklangen einzeln in
dem dunklen, leeren Raum.

»Nur das Erinnern,« sagte er auf einmal, und
es war ein seltsamer Klang in seiner Stimme, »das
Erinnern an jene Nächte ist nicht schwach, in denen
wir bei einer kostbaren Frau waren.«

Er brach ab.

Ich mußte wagen zu sprechen. »Die Nächte,«
sagte ich leis, »und du – du selbst?«

»Ja,« sagte er und lächelte.

[s 73]
Aus dem Frauensaal über uns kam leise ein­
töniger Gesang. »Sie schlafen noch nicht,« sagte er
und horchte auf. »Das ist Olpje, die singt.«

»Was ist es für ein Lied?«

»Es ist ein Lied von mir,« sagte er, »es handelt
vom Nebel.«

Und ich erinnerte mich auf einmal daran, daß
ich das Lied kannte. Ich hatte es gehört, als ich
das erstemal bei ihm zu Gast war. Mawrogeza
hatte es damals gesungen. Es handelte von einer
Frau, die auf einen Berg geht, durch Wald, über
Schnee, über Stein. Und der Nebel kommt zu ihr und
spielt mit ihrem leichten Gewand und spielt mit ihrem
Blick. Und erhebt sich um sie und senkt sich wieder
und lichtet sich und wird dichter und umhüllt sie
immerzu. Und gibt sie nimmer frei, auch auf dem
Gipfel nicht.

»Wir wollen das Feuer anmachen,« sagte der
Alte.

Wir zündeten den Holzstoß auf dem Feld hinter
dem Hause an, wo ihn die Knechte zu meinem
Abschied aufgeschichtet hatten. Wir saßen ein paar
Schritte davon und blickten in die Flammen, die
immer höher schlugen. Die Frauen kamen aus dem
Haus und setzten sich zu uns. Und nach einer
Weile begannen sie zu singen – Kewjy, die an ihren
[s 74]
Freier dachte, während sie sang, und Tero mit ihrem
schrillen Kinderstimmchen und Olpje, die Drossel;
und tief unter allem Mawrogezas Stimme, dunkel
und voll.

»Jetzt mußt du singen,« sagte Tero, als sie ge­
endet hatten und rieb mir lachend Pontroblüten in
die Hand. Alle lachten.

Ich konnte nicht singen.

»Dann mußt du uns etwas erzählen,« sagte
Mawrogeza.

Aber ich konnte nichts in ihrer Sprache erzählen.

»Erzähle uns in der Sprache deines Landes,«
sagte Olpje leis.

Da begann ich in der fremden Sprache zu erzählen,
und ich sprach wirr von dem, was mir gerade im Sinn
war. Sie starrten in die Glut, die am Vergehen war,
und horchten still den unverständlichen, harten Worten.
Das Feuer war erloschen, als ich schwieg.

»Was war es?«[|,] fragte Mawrogeza.

Ich antwortete ihr nicht.

Aber Olpje sagte auf einmal vor ihrem Groß­
vater und vor Mawrogeza und vor den Schwestern:
»Du liebst eine Frau in deinem Land.«

Ich lachte.[ –“
|“
<a>]
Es war dunkel geworden im Zimmer. Draußen
rauschte der Nachtwind durch die Bäume des Gartens.
[s 75]
– „Gehen Sie wieder einmal dorthin zurück?“ fragte
Frau Marie auf einmal. Ihre Stimme klang leis, als
schwinge nur eine Saite im Wind.

„Nein,“ sagte Hans Jung, und dann senkte er
den Kopf tief und sprach so leis, daß er später nicht
wußte, ob sie seine Worte hatte verstehen können:
„Für mich allein – gehe ich gewiß nimmer zurück.“

„Ich will Licht holen,“ sagte sie und stand auf.
Sie ging hinaus, ohne das elektrische Licht aufzu­
drehen. Hans Jung blieb im Dunkeln sitzen, bis sie
mit der Lampe wiederkam. Sie blieb unter der Tür
stehen. „Ich muß Fritz zu Bett bringen. Wollen
Sie mitgehen und zuschauen?“ Der gelbe Schein der
Lampe fiel auf ihren hellen Hals und auf die graue
Seide ihres Kleides, das ihre Brüste nicht heben sollte.

<s 0.20>Er folgte ihr in das Schlafzimmer ihres Kindes,
in dem grell das elektrische Licht brannte. „Ich bringe
ihn selbst zu Bett,“ sagte sie zum Dienstmädchen,
das den Buben entkleidete. Er stand mit seinem
langen, weißen Nachtkleidchen vor dem Bett und
lachte über das Mädchen, das vergebens versuchte,
ihm die Strümpfe von den Beinen zu ziehen. „Lump,“
sagte Frau Marie und nahm ihn auf den Arm.
„Lösche das Licht aus,“ sagte sie zu dem Mädchen,
das Hans verlegen anlachte, als es ging. Nun war
nur noch der gelbe Schein der Lampe in dem kleinen
[s 76]
Zimmer. Hans Jung lehnte an dem Kopf des weißen
Bettchens und sah zu, wie sie ihren Buben in die
Kissen hob und zurechtdeckte und ihm die Haare
strich, ehe sie sich wandte. – „Gute Nacht,“ sagte
Hans Jung und nahm die kleine Hand, die auf der
Decke lag. – „Gute Nacht,“ klang es ganz scheu
aus den Kissen, die im Dunkeln zurückblieben.<s 0>

Sie gingen wieder in das Erkerzimmer vor.

„Nun muß ich wohl fort,“ sagte Hans Jung.
„Nein,“ sagte sie und stellte die Lampe auf den Tisch
im Erker. Er hatte vergessen, sie ihr abzunehmen.

„Arbeiten Sie immer am Abend?“[|,] fragte sie.

Nein, er arbeitete nimmer seit den letzten Tagen.

„Was tun Sie am Abend?“

„Nichts,“ sagte er und sah sie gerade an. „Nichts.“

„Am Fenster stehen und auf den See hinaus­
sehen,“ sagte er.

Das Mädchen kam und brachte den Tee. „Wir
brauchen nichts mehr,“ sagte Frau Marie, „du kannst
schlafen gehen,“ und Hans Jung sah, wie sich das
rote Bauerngesicht unter der Tür zu einem dummen
Lachen verzog.

„Kommen Sie,“ sagte die Frau, „wir wollen uns
ins Dunkel setzen, und Sie sollen erzählen.“

Sie gingen zum Sofa hinüber und ließen die
Lampe auf dem Tisch zurück.

[s 77]
„Ich werde heute noch heiser werden,“ sagte
Hans Jung. Aber der lustige Ton paßte nicht zu
seinen Worten: „Es ist wahr, daß wir Tage haben,
wo wir uns packen wie einen Krug und uns um­
drehen und verschütten.“ – Aber dann senkten sie
zusammen den Kopf in der gleichen Art und sahen
vor sich hin und regten sich nicht, während <l schnee2>seine
Worte langsam niedersanken, tonlos wie Schnee in
einer langen Winternacht.

Er erzählte:

<l china_anfang><s 0.20>„Einmal in China hatte ich kurze Zeitlang einen
Freund, mit dem ich ein paar Wochen wanderte.
Es war ein großer, schlanker Mensch mit den schönsten
Augen, die ich bei einem Mann gesehen habe, wir
liefen zusammen den gelben Fluß entlang, von Dorf
zu Dorf, zu einer Zeit, wo alle Bäume weiß und
rot von Blüten waren und die Büsche an den Wegen
dufteten und die Frühlingsvögel in den hellen Nächten
schrien. Aber ich hatte ihn in einer Schenke in
Shanghai gefunden, in einem jener versteckten Häuser,
wo man in manchen Nächten solche Europäer sieht,
die ihre Straße verloren haben. Wir lagen auf der
gleichen Matte im unteren Stockwerk, und ich mußte
unverwandt die fremden blauen Augen ansehen, die
tief in den faltigen großen Höhlen lagen.<s 0>

Dicke Schwaden des süßen Rauches lagen damals
[s 78]
über unserem kleinen Saal. Denn es war schon spät;
die meisten Gäste schliefen. Rings um uns lagen
Chinesen. Nur auf einem Polster in der Ecke, uns
gegenüber, lag einsam ein Abendländer, ein alter
Mann, der einen zerrissenen, schäbigen Gehrock trug.
Er wälzte sich unruhig von einer Seite auf die andere
und sang von Zeit zu Zeit immer wieder die gleichen
Worte, einen kleinen Teil eines Liedes; ich hörte,
daß seine Sprache slavisch war. Manchmal erhob
sich einer der Chinesen und taumelte wirr zur Tür
ins Freie. Nur noch selten kamen neue Gäste, sie
kamen mit leiser Musik und gingen zur Treppe, die
ins obere Stockwerk des Hauses führte. Aus einem
Zimmer über uns drang ab und zu der eintönige
Sang einer tiefen Frauenstimme herab.

»Sie wollen Verkommenheit beobachten, wie?«[|,]
sagte plötzlich der Fremde neben mir auf deutsch.
»Lasterhöhlen studieren, nicht?«

Er sah mich an, ein wenig spöttisch und ein
wenig gereizt. Ich hörte aus seiner Aussprache, daß
er ein Skandinavier war.

»Sind  S i e  aus diesem Grunde hierhergekommen?«[|,]
antwortete ich.

»Nein,« sagte er kurz und legte sich wieder
zurück, ohne mich weiter zu beachten.

Eine alte, zerlumpte Frau kam an unser Polster
[s 79]
und ließ sich eine Pfeife geben. Sie lachte uns zu
und sagte ein paar chinesische Worte, die ich nicht
verstand. Aber der Fremde warf ihr eine Antwort
hin, daß sie den Kopf duckte wie vor einem Schlag.
Dann starrte er wieder in die Luft und schwieg.

Ich weiß nicht, warum ich es tat. Ich legte mich
auf einmal zurück, daß ich näher neben ihn zu liegen
kam und sagte unvermittelt: »Ich bin aus dem gleichen
Grunde hier wie Sie; ich bin hier, um hier zu sein.«

<s 0.20>Er lächelte nur, und es schien mir, als müsse er
sich erst auf den Inhalt der fremden Worte besinnen.
Aber auf einmal lehnte er sich empor und lachte auf
und sagte: »Das ist auch der einzige Grund, warum
wir auf der Welt sind; wir sind auf der Welt, um
auf der Welt zu sein.«<s 0>

<s 0.18>Er legte sich wieder zurück und blieb still. Aber
nach einiger Zeit weckte er die Chinesin, die neben
ihm zurückgesunken war und begann mit ihr zu
streiten. Ich verstand wenig von dem, was er sagte.
Aber es mußten die häßlichsten Schimpfworte sein,
denn die Frau verzog entsetzt ihr berauschtes Gesicht
und antwortete immer mit dem gleichen gräßlichen
Fluch, den ich zufällig kannte. Da lachte er auf und
es schien, als habe er nichts anderes gesucht, als die
Lust, verflucht und beschimpft zu werden. Aber dann
sagte er etwas, daß sie sich jammernd erhob und
[s 80]
auf ein anderes Polster wankte, wo sie sich neben
zwei jungen, gutgekleideten Chinesen, die mit weit
offenem Munde schliefen, niedersinken ließ.<s 0>

<s 0.20>»Jetzt sind wir sie los,« sagte der Fremde und
lachte.<s 0>

Ich betrachtete seine Hände, die dicht neben
meinem Gesicht lagen. Sie waren schmal und schlank
und ein wenig gebräunt. Sie sahen aus, als seien
sie gut dazu, über das Haar einer kostbaren Frau
zu streichen.

Sein Kopf war tief im Schatten, und ich konnte
nicht sehen, ob er die Augen offen oder geschlossen
hielt. Nur das kleine Kinn war noch von dem roten
Schein der Papierlaterne gestreift, die über uns hing.
Aber ich hätte nicht sagen können, ob er zwanzig
oder vierzig Jahre alt war . . .

Eine junge, schlanke Chinesin kam die Treppe
herab, Schritt für Schritt, lässig und stolz. Auf der
untersten Stufe stand sie still und sah lächelnd über
den schlaftrunkenen Saal.

Der Fremde erhob sich und rief ihr ein paar
Worte zu, deren Sinn ich nicht verstand.

Da senkte sie den Kopf und schlich zur Tür,
die auf die Gasse führte.

Aber sie mußte an unserem Lager vorbei. Er
stand und wartete, bis sie kam. Er nahm sie beim
[s 81]
Handgelenk und hielt sie ein wenig fest und gab ihr
einen kleinen, blauen Blütenzweig, mit dem er gespielt
hatte. Da nahm sie seine Hand und legte sie auf
ihre flache junge Brust.

Dann ging sie.

Er sah ihr nach, bis sie verschwunden war und
legte sich wieder aufs Polster zurück.

Und auf einmal lachte er auf, kurz und sinnlos.

»Was haben Sie ihr zugerufen?«[|,] fragte ich leis.

»Etwas[, –«| –«,] er suchte nach dem rechten Wort
und sagte dann: »Etwas Unrichtiges.«

Ich fragte nicht mehr, und er sagte selbst nach
einer kleinen Weile:

»Ich sagte, daß dieses Haus ein schlechter Altar
für ihre Brüste sei.«

<s>Er sprach es schnell und in einem Ton, als sei
ihm nichts daran gelegen, daß ich ihn verstehe. Aber
ich antwortete ihm doch, vielleicht weil ich den schwer­
mütigen Blick gesehen hatte, mit dem seine nordischen
Augen dem Mädchen nachblickten, als es ging; ich
sagte: »Ob sie Ihre Worte recht verstanden hat?«

»Verstanden hat sie es nicht,« sagte er, »aber
es kann sein[, –«| –«,] er suchte wieder nach einem Wort,
»es kann sein, sie hat es empfunden.«

Aus dem Zimmer über uns klang melancholisch
wieder die tiefe Frauenstimme herab. Es war das
[s 82]
gleiche Lied und, wie mir schien, die gleichen Laute
wieder.

»Das ist Pa=Yo,« sagte er, »aber es wird niemand
in ihr Zimmer kommen, um Tee bei ihr zu trinken.
Sie ist alt geworden.«

Der Sang brach ab.

Er erhob sich vom Polster und reckte sich.

»Trinken Sie noch einen Tee mit?«[|,] fragte er.
Ich folgte ihm die Treppe hinauf.

<s>Wir waren beide nüchtern, nüchtern und wach.
Aber die Gäste, die in der Diele des oberen Stock­
werks lagen, zwischen den kleinen Chinesenmädchen,
die sie bedient hatten, waren voll Schlaf und Traum.
Auch hier war wieder eine Matte mit Europäern,
drei jungen Männern, die sich in unruhigem Schlafe
wälzten und französisch fluchten. »Es sind Seeleute,«
sagte der Frem­de.

»Soll uns Yü=Süh den Tee machen?«[|,] fragte er.
Er ging zu einer der Türen, die auf den Saal mündeten
und öffnete sie. »Kommen Sie,« sagte er.

Yü=Süh schlief auf einer Matte unter der kleinen
Papierlaterne, zusammengerollt wie ein Bündel von
roter und grüner Seide. Der Däne weckte sie. »Tee,«
sagte er. Wir lagerten uns auf der großen Gast­
matte, die in der Mitte des Zimmers lag. Yü=Süh
arbeitete lautlos an dem elfenbeingezierten Tee­
[s 83]
tischchen, mit Händen, die schlank und geheimnisvoll
aus dem weiten Ärmel ihres Kleides kamen.

Ich unterbrach das Schweigen, das über uns
gesunken war.

»Yü=Süh ist vollkommen schön,« sagte ich.

Er lachte ein wenig, ehe er fragte: »Wie lange
Zeit sind Sie schon in China?«

»Fast ein Jahr,« sagte ich.

»Glauben Sie, daß Sie schon sehen können,
wann eine Chinesin vollkommen schön ist?«

Yü=Süh sah abwechselnd auf die beiden Münder,
aus denen die unverständlichen fremden Worte kamen.

»Sie ist schön, es ist wahr,« sagte er und sah
sie an.

Später rief er ihr etwas zu, wovon ich nur das
Wort »Augen« verstand.

»Wir wollen uns deutsch unterhalten,« sagte er,
»Sie sind doch ein Deutscher, nicht? – Yü=Süh soll
uns heute nur mit den Augen zuhören.«

Und er begann auf einmal zu erzählen von sich
und von den Wegen, die er ging, und er erzählte
wie Menschen, die immer einsam sind und denen
nichts daran gelegen ist, sich ganz vor anderen zu
entblößen[,|] – weil sie wissen, daß sie wieder von
ihnen gehen und wieder allein sein werden.

Yü=Süh hörte mit den Augen zu.

[s 84]
Und manchmal stockte er und suchte nach dem
rechten deutschen Wort, und manchmal hob mitten
unter seinen Worten Pa=Yo in dem Zimmer nebenan
ihren kurzen Gesang an. Niemand beachtete es mehr.
Aber er erzählte im Grunde nur immer das gleiche,
und er variierte das, was er sagte, wie man ein
musikalisches Thema variiert . . . daß er in einem
kleinen Fischerdorf an der Nordsee geboren sei und
jetzt hier sitze bei Yü=Süh, die ihre Augen aufsperre
wie einen schwarzen Brunnen.

Auf einmal merkte er, daß der schwarze Brunnen
geschlossen war und daß Yü=Süh schlief. Da lachte
er auf, daß es durch das Haus gellte, in das eine
große Stille gekommen war.

»Schlafen  S i e  noch nicht?«[|,] fragte er.

Nein, ich war wach – wach.

»Es wird bald Tag,« sagte er, »komm.«

Wir gingen, ohne Yü=Süh zu wecken und legten
ein paar Geldstücke neben sie auf die Matte.

Auf der Gasse dämmerte es schon. Die Häuser
waren still geworden, nur selten drangen noch Stimmen
aus ihnen; wirrer leiser Sang und gedämpfte Bambus­
geigen. Unsere Schritte hallten in der Stille wider,
und unsere Schatten liefen gespenstisch vor uns her.
Der Himmel war aus dunklem Stahl . . .

Und dann gingen wir ein paar Wochen wie
[s 85]
Freunde längs den Frühlingsufern des breiten Flusses
und horchten zusammen, wie die Zikaden und die
fremden Frauen in den Reisfeldern sangen.

Wir schwiegen viel, und wenn der Fremde sprach,
war es nur eine neue Variation eines alten Themas.

[. . |]»Was wollen wir werden?«[|,] fragte er.

Ich wußte, daß er auf keine Antwort wartete
und blieb still. Wir lagen rastend unter einem
Hängestrauch am Weg.

»Werden wir Arbeiter,« sagte er, »so versenken
wir unser Leben in eine Arbeit . . und es ist ver­
loren. . . Warum sollen wir es verlieren?«

Ich schwieg und sah auf den Fluß hinüber. Der
Wind trieb ab und zu einen schaumigen Kamm auf
die schwarzen Wogen. Sie rauschten eine schwere
Melodie.

»Und sind wir Künstler,« sagte er, »so ver­
senken wir unser Leben in unsre Kunst und ver­
lieren es dort. . .«

Und nachdem er ein wenig still gewesen war,
setzte er sich auf und sagte das, was er immer sagte[,|]
– was er sagte, als wolle er seinen Glauben daran
fester verankern:

»Und wenn Sie der Mann einer Frau sind,«
sagte er, [. . »|» ... ]es ist wahr, an den warmen Sommer­
abenden sitzen Sie mit ihr im Garten, und sie hat
[s 86]
[i|I]hr Kind auf dem Schoß und singt: Rururu, Rururu,
sloep min Peer – rururu, . . .«

Er brach ab. Ich hörte den Wind durch die
Blüten unseres Strauches gehen. Er lächelte traurig
und versonnen. Und das Lied des Flusses klang
mir, als risse es an lockeren Ankerketten.

»Aber Ihr Leben haben Sie in die Brüste der
Frau versenkt,« sagte er schwer, »und es ist ver­
loren.«

Und dann sprang er auf und stand mitten in
den Blütenzweigen des Hängestrauches, und seine
Worte verhallten leer über den Wogen:

»Ich will vielem Gast sein – und will nirgends
bleiben als bei mir.«[ –|]

Und später gaben wir uns die Hand und
trennten uns, und ich habe nichts mehr von ihm
gehört und werde ihn nimmer sehen.“[ –|]<l china_ende>

<s>Über die zwei gebeugten Köpfe und das halb­
dunkle Zimmer sank ein Schweigen, tief und dicht,
als könne es nimmer durchbrochen werden. Der leise
Gang der Uhr wob sich hinein und das Rauschen
des Windes, der die Büsche vor dem Hause bog.

Hans Jung hob seine Hand. Sie war ihm schwer,
als müsse er sie in eine tiefe Wunde legen. Er nahm
die Hand der Frau, die still in den Falten ihres
Schoßes lag, und hob sie zu seinen Lippen.
[|<aa>]
[s 87]
Sie begleitete ihn bis an das Gartentor, als er
ging, bis in den tiefen Schatten des Nußbaumes,
dessen Krone auf den Weg hinüberragte.

„Gute Nacht,“ sagte sie, ohne sich zu wenden.
Sie standen sich gegenüber und rührten sich nicht.
Es war eine finstere Nacht mit schwarzen, ziehenden
Wolken.

Da nahm er sie bei den Händen und zog sie
an sich, und sie lehnte sich schwer an seine Brust.

Dann entzog sie sich sacht und ging.

Hans Jung stand vor dem Tor und sah, wie
sie über den hellen Gartenweg schritt und in dem
Dunkel des stillen Hauses verschwand. Sie ging
langsam und gebeugt, wie ein Mädchen, wenn die
Nacht zu Ende ist.

Er wandte sich.

Er ging dem Wege nach, der vom Dorf zum
Wald hinüberführte, zum Wald mit den tausend
stillen Bäumen und zum Berg, der sich schwer in den
dunklen Himmel hob. Hundert Menschen gingen den
Weg am Tag, sprachen und lachten und betrieben
ihre Täglichkeiten. Jetzt war er still und leer, als sei
ihn der letzte Mensch vor vielen Jahren gegangen.
Der Wind trieb durch die Bäume, daß die Tropfen
fielen, die an den Blättern hingen. Er kam in kurzen
Stößen, gewaltsam, daß sich die dichtesten Kronen
[s 88]
bogen, und flog vorbei, irgendwohin, daß es stille
wurde, wo er gegangen war. Und dann kam er
wieder zurück, aus einer anderen Ferne.

Hans Jung ging bergwärts und hörte nicht den
Wind gehen und sah nicht die Kronen sich beugen.

Ihm schien es still – still wie in einer hohen
Kirche. Und das Lied, das durch ihre leere Halle
tönte, war von der Frau, von der er kam, . . und
war, er wußte nicht warum, von seinem Vater. . .

<l vater_anfang>Von dem stillen Mann, der gestorben war und
hatte Frau Marie nicht gesehen. . .[
|
\\
\\
\\
\\]
„Träum', Bub,“ sagte der Vater, wenn er an
manchen Abenden in das weiße Knabenzimmer kam
und Hans war schon zu Bett gegangen.

„Gute Nacht, Vater,“ sagte Hans.

„Wer nicht etwas Schönes träumt, ist dumm,“
sagte der Vater und lachte. Und ehe er ging, strich
er kurz über das Haar des Knaben, auch als Hans
schon größer geworden war, und es Nächte gab, in
denen er von Frauen träumte.

„Von was träumst  d u ,  Vater?“[|,] fragte Hans
eines Abends, ehe sich die Hand, die weich auf
seinem Haare lag, löste. Es war schon die Zeit,
wo er das alte Cello bekommen hatte, und es fiel
ihm mitten im Satze ein, daß er eine freche Frage
an den verschlossenen alten Mann tat.

[s 89]
Aber der Vater lachte: „Ich – ich träume von
meinen Träumen,“ sagte er. Hans konnte es nur
dunkel verstehen.

Der Vater ging hinüber in das große Eck­
zimmer und setzte sich an die grünbelegte Platte
seines mächtigen Pultes. Er blätterte in Urkunden
und Scheinen, schob die Fächer seiner Schränke auf
und zu und schrieb und strich und rechnete – rechnete
immerzu, in Zinsfüßen und Zinsen und Zahlen.
Aber es waren nicht mehr die winzigen Zahlen aus
der ersten Zeit seiner Arbeit, aus der ärmlichen Zeit,
wo er mit kleinen Leuten hatte wuchern müssen.
Nun waren es großstellige Zahlen und große Beträge
geworden.

<s>Manchmal lehnte er sich in seinen Sessel zurück
und ruhte. Und er lächelte dabei über das Heer der
Zahlen – das nichts war als ein Spiel für seine
ruhelosen Hände. Dann war der Kopf tief im Schatten
des grünen Lampenschirmes, die schlanke, fein­
gekrümmte Nase und die Augenhöhlen mit den
großen buschigen Brauen; und der Mund, der
stundenlang starr und unbeweglich bleiben konnte,
mit dem leichten Zug vornehmer Zurückhaltung in
den Winkeln; und der gefürchtet gewesen war von
den kleinen Leuten, damals, als das Geld noch kein
Spiel für ihn gewesen war, als er es verdienen
[s 90]
mußte für eine Frau; der Mund, der lächeln konnte,
daß der Knabe den ganzen Tag über sang, im
Garten und auf den breiten Treppen des großen
kalten Hauses. Und er wußte nicht, woher sein
Singen kam.

„Magst du nicht ein wenig spielen?“[|,] sagte der
Vater nach dem Abendessen.

Hans ging in sein Zimmer hinauf, um das
kleine, rotlackierte Instrument zu holen, auf dem er
lernte. Aber es stand nicht an seinem Platz neben
dem kleinen Bücherschrank. Es war nirgends zu finden.

„Vielleicht hast du es im Musikzimmer stehen
lassen,“ sagte der Vater, als er zurückkam. Nein,
Hans spielte nur selten im Musikzimmer. Dort stand
ein Harmonium und ein Flügel und niemand spielte
darauf. Es war noch von der Mutter her, die ge­
storben war.

„Du hast es vielleicht bei Hegel gelassen,“ sagte
der Vater. Hegel war sein Lehrer, ein dicker Mann
mit einem weißen Bart. Er hatte einen berühmten
Namen, obgleich er nur wenig in der Öffentlichkeit
spielte. Wenn seine Cellostunden bis in die Däm­
merung reichten, zündete er kein Licht an, und das
Ende seines Unterrichts fiel im Winter oft in tiefe
Finsternis. <l kleiner_kasten>„Erst wenn du auf nichts mehr achten
mußt, während du spielst, und mit deinem kleinen
[s 91]
Kasten in den Wolken herumfliegst, kannst du
etwas,“ sagte er zu Hans.

„Oder du hast es bei Lore Hansen gelassen,“
sagte der Vater und verzog keine Miene zu seinen
Worten. Lore Hansen war die Schwester von Gott­
fried Hansen, seinem Freund, und der Vater hatte
ihn vor ein paar Tagen mit ihr gehen sehen. Er
war mit dem schlanken Mädchen durch das Ge­
dränge der Hauptstraße gegangen und hatte ihr von
seiner letzten Ferienreise erzählt, von den Menschen
in den großen Hotels, in denen er mit dem alten
Mann gewohnt hatte, von der Nordsee und ihren
Dünen und ihren Fischerdörfern und Stürmen. Und
auf einmal war vor ihnen das ernste Gesicht des
Vaters aufgetaucht, mitten unter den vielen anderen
Gesichtern, die vorüberglitten. Er hatte zuerst Lore
Hansen und dann seinen Sohn angesehen, klar und
kurz, und dann seinen breiten schwarzen Hut ge­
zogen und sie gegrüßt wie ein guter Bekannter.

„Nein,“ sagte Hans und sah ihn offen an[|,] „ich
habe sie nicht mehr gesehen, seit wir dir begegneten.“
Er log; er war vor zwei Stunden mit ihr in den
Flußanlagen vor der Stadt gewesen und hatte sie
zum erstenmal geküßt.

„Ich habe es dir nicht gestohlen,“ sagte der
Vater mürrisch, „gute Nacht.“ Und als Hans bei
[s 92]
der Tür war, rief er ihm noch nach: „Es handelt
sich nicht um den Verlust, sondern um den natür­
lichsten Ordnungssinn.“

Aber als Hans verärgert und traurig an dem
Fenster seines einsamen Zimmers stand, kam der
Diener, um ihn wieder herabzurufen. Und als er
trotzig in das Zimmer zurückkam, hatte der Vater
das dunkelbraune kostbare Cello neben seinem Stuhl
stehen.

„Es ist alt,“ sagte er nur – mit einem Lächeln,
in dem der Knabe dunkel die Scham fühlte, die der
Vater über die kleine Sentimentalität seines Schenkens
empfand.

„Spiel'“, sagte er und schlug ihm leicht auf den
Mund, als er ihm die Hand küssen wollte.

Hans spielte ein paar alte Tänze, die er aus­
wendig wußte, und die der Vater gerne hörte:
Rondo, Menuett und Gavotte. Und das Cello sang
schon dort, daß der alte Mann den Kopf auf die
Brust sinken ließ und vergaß, daß er nicht allein war.

Aber als Hans in sein Zimmer gegangen war
und das köstliche Geschenk verwahrt hatte, vergaß
er auf einmal den Vater und Lore Hansens jungen
Mund. Er dachte an die vielen Orte, wo er noch
auf dem Cello spielen würde. Und an die Frauen,
denen es singen sollte, und die noch in tiefem Dunkel
[s 93]
vor ihm lagen. Und er hatte noch nichts gewußt
von Frau Marie –[ . . .
|
<aaa>]
Dann kam die Zeit, wo ihn Hegel, der immer
noch [»du«|„du“] zu ihm sagte, nimmer unterbrach und
oft still saß, als die Töne schon lang verklungen
waren und nur sagte: „Ich danke dir.“

Aber am liebsten spielte er beim Vater. Sie
trafen sich fast nur zu den Mahlzeiten und dabei
wurde viel geschwiegen. Und als Hans älter wurde,
hatte er begonnen, die undurchschreitbare Fremdheit
zu fühlen, die den alten Mann umgab. Sein Lächeln
und seine Geschenke waren nur eine Maske dafür.

Aber an den seltenen Abenden, wenn das Cello
gesungen hatte, daß das Zimmer ins Blaue versank,
konnten die Augen des alten Mannes blicken, wie
sie waren.

„Woher hast du es?“[|,] fragte er in jenem Ton, der
Hans Jung durch die Brust ging, wann immer er kam.

Er wußte nicht, woher er es hatte. Er sprach
ja nie mit dem Vater von den Wegen, die er ging.
Wenn es so weit kam, daß er davon erzählen könnte,
schien ihm beim tiefen Klang jener Stimme alles
nichtig, was er selber mit sich trug. Was hätte er
erzählen können?

<l lore_hansen>Von Lore Hansen – daß er eines Tages ge­
merkt hatte, wie dumm und kleinlich sie war. Und
[s 94]
er hatte fortgefahren, mit ihr durch die Wiesen zu
gehen und weiter an sich glauben zu lassen, wie ein
Schuft. Er hatte es getan, weil es dennoch angenehm
war, seinen Kopf still an die kleinen Mädchenbrüste
zu legen und zu horchen; und weil das Voneinander­
gehen von selbst kommen mußte.

Oder von dem Schulfreund, mit dem er immer
zusammen gewesen war beim Rudern und Tanzen­
lernen und Reiten und später hinter den Mädchen
her. Und der eines Tages in einem heftigen Streit
alles Gift ausgab, das sich in den langen Jahren still
gesammelt hatte. – „Wucherersohn,“ rief er. Und
Hans Jung schlug ihm die Fäuste ins Gesicht, daß
er aufschrie und vergaß, sich zu wehren, er warf ihn
zu Boden und schlug blind auf ihn ein, bis er sah,
daß er blutete. Es wäre ihm wie eine schmutzige
Berührung für den Vater vorgekommen, wenn er es
nur erwähnt hätte. . .

<s>Und von dem Fest zu erzählen schämte er
sich – von dem Fest bei der fremden Frau, zu der
ihn ein Schüler Hegels, Istritsch, der Russe, eines
Nachts geführt hatte. Scheu wie ein Mädchen war
er in den halbdunklen, kleinen Saal getreten, in dem
sich eine weiche Musik um die festlichen Kerzen
und die warmen Körper der Frauen wiegte, die ihre
Hüften und Brüste zeigten, wenn sie standen und
[s 95]
gingen. Denn sie waren alle in den gleichen, ver­
wirrenden Gewändern dünner Seide, die nichts waren
als ein Schmuck ihrer nackten Glieder. – „Du hast
Augen wie Sonne auf dem Meer,“ sagte die Frau,
mit der er tanzte, und schmiegte ihre weißen Brüste
an seine fiebernde Haut. Seine Hände tasteten
über ihren schlanken Rücken. „Aber du bist reicher
wie das Meer,“ flüsterte er. Um Mitternacht zog
sie ihn aus dem Saal und führte ihn in das obere
Geschoß des fremden Hauses, auf dessen Treppen
stille, japanische Laternen glommen. Und dunkel
und warm und voller Stille war das Gemach, wo
sie sich ihm gab. – „Sag' etwas,“ flüsterte sie tief
in der Nacht und strich ihm übers Haar. Und er
wußte, daß er am Tage fremd an ihr vorübergehen
würde, als er sie küßte und sagte, weich und leis:
„Du bist die erste Frau. . .“ Das Morgenlicht schlich
wie fahler Nebel durch die Gardinen, als er aufstand
von dem Frauenbett und ging . . „Woher ich es
habe,“ sagte Hans lächelnd. Er stellte das Cello
zur Seite und setzte sich zu dem Vater an den Tisch.

„Triffst Du Lore Hansen noch?“[|,] fragte der Vater
in gleichgültigem Ton. Es war das erstemal seit
einem Jahr, daß von dem Mädchen gesprochen wurde.

„Nein,“ erwiderte Hans. Er lachte kurz auf
und sagte: „Sie ist ja dumm, furchtbar dumm.“

[s 96]
Der Vater lächelte leis.

„Ist sie schon vorbeigegangen,“ sagte er und
sprach es wie den Vers eines Gedichtes. Und dann
sagte er mit stillem Sinnen zwischen den einzelnen
Worten und mit einem Klang in der Stimme, daß
Hans Jung hätte zu Boden sinken mögen, während
er horchte:

„Ich möchte gern die Frauen sehn, . . zu denen
du noch kommen wirst.“

Hans wagte nicht zu atmen, bis der alte Mann
weiter sprach:[
|]
„Wie sie gehen und wie sie sind.“

Er brach ab, und der Kopf sank ihm auf einmal
schwer auf die Brust[. . .
|.
<aaa>]
Und dann war er gestorben und hatte Frau
Marie nicht gesehen. . .

Es war in dem Jahr, in dem Hans in Paris
war. Der alte Haus­arzt hatte ihm geschrieben, und
er war bang und erschrocken durch den November­
regen und das kah­le Land zu­rück­ge­fah­ren – und
doch ein wenig unwillig über die Störung.

„Soll ich mit dem Herrn gehen?“[|,] fragte sein
junger Pariser Diener, als er ihm den Koffer packte.

„Nein,“ sagte Hans, „machen Sie sich vergnügt,
bis ich wiederkomme.“

„Danke,“ sagte der Diener.

[s 97]
„Und Sie brauchen nichts zu tun als jeden Tag
einen Strauß Rosen zu Frau Boucher zu tragen.
Kaufen Sie sie bei Camille; gelbe, langstielige, bitte.“

„Nimmt der Herr das Cello mit?“[|,] fragte der
Diener.

„Danke, daß du mich daran erinnert hast,“
sagte Hans und schenkte ihm ein Goldstück.

Nun konnte er dem Vater vorspielen, wenn die
Dämmerung gekommen war und das Morphium die
Schmerzen genommen hatte.

„Geben Sie ihm Morphium, soviel Sie dürfen,“
sagte Hans zu dem Professor.

„Kopf hoch, junger Mann,“ sagte der Professor.

Hans lächelte.

Aber außer jenen stillen Stunden, in denen er
allein bei dem Sterbenden saß, glitt alles an ihm
vorbei wie Schattenbilder; die vielen liebenswürdigen
Menschen, die auf einmal um ihn waren, der däm­
mernde Spätnachmittag, an dem er hinter dem langen
Sarg durch die dichten Schneeflocken schritt, die auf
den nassen Friedhofwegen vertauten.

Es war eine alltägliche Beerdigung, mit ein
paar Herren in vornehmen Mänteln und Zylindern,
Bekannte von der Börse; und mit Hegel, der so alt
geworden war, daß er kaum mehr gehen konnte,
und der zu Hans, als er ihm die Hand gab, nur sagte:
[s 98]
„Laß es dir immer gut gehen, Hans,“ und als die
Zeremonie vorüber war, fuhr er in der alten Equipage
des Vaters nach Hause, einsam auf den rotgepolsterten
Sitzen und dachte an nichts Feierliches gerade; <l beerdigung>er
dachte zufällig an das, was die schwarzen Figürchen,
die jetzt unter ihren Regenschirmen aus dem Friedhof
kamen, wohl miteinander sprechen würden. – „Ist
der Sohn der einzige Erbe?“ – „Ja.“ – „Er will
Künstler werden?“ – „Er kann es sich leisten.“ –
„Ist es nicht unheimlich.“ – „Was?“ – „Alle
Wucherer sterben an Krebs.“ – Hans Jung lächelte
vor sich hin, während der Schnee an die kleinen
Wagenfenster fiel und schmolz, daß die Lichter der
Straßen tausendfach gebrochen wurden.

Und erst Wochen später fielen ihm Bruchstücke
aus den letzten Gesprächen des Sterbenden ein, und
auf den Sinn mancher Worte kam er erst nach Jahren.

„Spiel',“ sagte der Vater, wenn er am Abend
erwachte. Hans stand von seinem Stuhl am Bett
auf und spielte. Aber wenn er zu den geschlossenen
Augen hinübersah, wußte er nie, ob der Vater ihn
hörte. Die hohen Kerzen des alten silbernen Leuchters
brannten neben dem Bett und flackten leise Schatten
um die mächtigen schwarzen Möbel. Der Diener
im Vorzimmer war auf seinem Stuhl zusammen­
gesunken und schlief, daß man seine Atemzüge bis
[s 99]
in die Stille des Gemachs herüberhören konnte, wenn
das Cello schwieg. Wenn Hans gespielt hatte, trat er
leis ans Bett zurück.

Es ging ein stilles Lächeln über das Gesicht
des Vaters.

„Gibt es noch Frauen in Paris?“[|,] sagte er und
bewegte nur langsam die bleichen Lippen.

„Frauen[,|] – daß man glaubt, man habe nur
geschlafen, ehe man sie sah. . .“

Er hatte die Augen geschlossen, und seine Lippen
blieben offen, wenn die Sätze erstorben waren. Es
sah aus, als tränke er.

„Ehe man sie sah . . und wie sie sind und durch
das Zimmer gehen . .“

Die Fenster rauschten unter dem Abendregen,
den der Wind durch die leeren Straßen trieb.

„Regnet es?“[|,] fragte er und wandte den Kopf
in die Kissen zurück, daß sich die Schatten über die
abgemagerten Züge legten.

„Im Grund – im Grund fließt es in das gleiche
Nichts, ob wir, was wir gesammelt haben, in uns
verschließen, oder ob wir es verschenken. . . Darum
sollen wir es lieber verschenken, mein Bub[.

. . |. .

]Und dennoch gibt es Stunden, für die alle
Leiden und Beklemmungen, die wir zahlen, zu gering
sind . . alles wird schwach neben ihnen, ihr Bild
[s 100]
allein verblaßt sich nicht . . die Nächte, wo wir bei
einer kostbaren Frau waren. . .“

Seine Worte erstarben.

Hans hörte seine eigene Stimme wie einen
unwürdigen Klang, der durch die Stille eines hohen
Tempels geht, und dennoch wagte er zu sprechen.

„Und du,“ sagte er, „du selbst?“

Es ging ein Lächeln über die verfallenen Züge.

„Hol' mir die Blätter, die auf dem Boden der
silbernen Schatulle liegen,“ sagte der Vater, „auf dem
gelben Tisch.“

Hans ging in das Schreibzimmer hinüber, wo
der eingelegte gelbe Tisch stand. Die Schatulle war
offen, früher war sie verschlossen gewesen, er wußte
es. Denn an manchen Tagen, wenn er sicher war,
nicht überrascht zu werden, hatte er sich in das stille
Eckzimmer geschlichen und hatte die Türen und
Deckel versucht, hinter denen die Geheimnisse des
alten Mannes verschlossen waren. Nie hatte er
etwas gefunden.

<s>Es waren Briefe in der Schatulle, hunderte, und
alle auf dem gleichen grauen Papier. Auf dem Boden
lagen ein paar lose geheftete Blätter in einem schwarzen
Umschlag – Gedichte. Er nahm sie heraus und
ging ins Zimmer zurück. Der Diener im Vorzimmer
hörte ihn nicht; man sah nur den tiefgesenkten Rücken
[s 101]
des Schlafenden. Der Vater hob langsam die Lider,
als er zurückkam.

„Hast du es gefunden?“[|,] fragte er. Seine Augen
sahen klar aus den tiefen Höhlen.

„Lies noch nicht,“ sagte er, „sie sind für deine
Mutter.“

<s>Und er schloß die Augen wieder und lächelte,
ehe er weitersprach: „Einmal ging ich mit ihr in einem
lichten Wald. Ich war noch arm damals. Und ich
hatte sie noch nicht geküßt. Es war eine helle Nacht,
und die Bäume ragten in die Höhe, wie ich es nimmer
in meinem Leben sah. Da sagte ich, ich weiß die
Worte noch: Ich will ein Haus verdienen, wo eine
Frau auf dem Altane sitzen kann und in die Wiese
sehn, zwischen denen Rosen und blühende Sträucher
stehn; und in die Nacht, die kommt.“

Die alte Hand auf der weißen Decke hob sich
ein wenig und sank wieder zurück.

„Und dann habe ich verdient und gestohlen,
lachend gestohlen, bis wir reich waren . .

Und dann kam die Zeit, wo ich sah, daß ich
sie nur liebte, um etwas zu haben, für das ich schaffen
konnte. Und daß sie nichts war als eine Wand
für meinen Schatten . .

Und dann hat sie es selbst gesehen . .

Und wenn die Frauen merken, daß sie nichts
[s 102]
sein können, als der Krug, der still wartet auf den
Gast – –“

Er vollendete nicht seinen Satz und wandte den
Kopf und schwieg. Erst nach einem Schweigen, das
Hans wie ein ewiges Warten schien, sagte er: „Aber
dann ist sie jung gestorben, als sie dich bringen
sollte . . es sind alte Lieder für sie. Lies.“

„Wie deine Schrift sich geändert hat,“ sagte Hans.

„Lies,“ sagte der Alte und schloß die Augen,
die er kurz geöffnet hatte. Hans begann zu lesen,
den Kopf tief über die fremden Sätze gebeugt, und
die Stimme war ihm am Versagen. – „Weiter,
weiter,“ sagte der Vater, wenn er stockte. „Lies
weiter.“ Da las er das nächste Lied.

<v begin 7.5cm>
Mein Zimmer ist still und leer.\\*
Ich weiß, daß du nicht kommen wirst.\\
Über die grauen Vorstadthäuser\\*
ziehn die Wolken träg und schwer.

Ich warte. Bald wird es schnei'n.\\*
Die Blüten mit den tausend Düften,\\
die Sommernächte werden kommen,\\*
und du wirst nicht gekommen sein.

Wenn ich mitten im weiten Meere bin,\\*
und <n mein Boot liegt im Streifen der Abendsonne,>\\
und wenn ich auf hohen Bergen bin,\\
[s 103]
und unter tausend Gletschern\\*
solltest du bei mir sein.

Nein, nein – du sollst mich nicht besuchen,\\*
wenn ich in Meeren und auf Bergen bin[, |]–\\
in jenen langen Nächten, wenn um mich\\
das Nichts sich lagert, namenloses Nichts,\\*
solltest du bei mir sein.

Ich kann in deine klaren Augen sehn\\*
und dir Rosen schenken\\
und den Blick leis senken\\*
und stille Stunden dir zur Seite gehn,

kann die Narren verlachen tausendmal,\\*
die im Nebel spielen\\
und nach Götzen schielen\\*
und sich drängen in ihrem engen Tal.

Und kann dir Lieder von Blut und Lüsten\\*
und von blauen Dingen\\
und tiefen Meeren singen,\\*
und wie wir unsere Tränen küßten –

Und kann dir nie und bis ins Grab\\*
nie sagen, wie ich lieb dich hab'.

Lieg' still . . die Welt versank\\*
in deine sommerlichen Brüste\\
und in das Meer der Lüste,\\
[s 104]
das ich aus deinen Augen trank.\\*
Warum weinst du . . .

Warum weinst du . . .\\*
Da ich nicht rasten will,\\
eh' deine Tempel stehn.\\
Hörst du den Brunnen gehn\\*
vor unserm Haus . . . Lieg' still.

Denk' dir, ich wußte nicht, daß du bist,\\*
und konnte Farben sehn,\\
und wußte nicht, daß du kommen wirst,\\*
und konnte die Straße gehn.

Und dann hab' ich dich gefunden,\\*
dich und die festliche Frist,\\
da ich Buhle war in dem Tempel,\\*
der um deine Schritte ist.

Und werde von dir gehn,\\*
irgendwohin, allein,\\
und die Menschen werden weiter\\*
in die große Stille schrein.

Als sie sich wandte, von ihm zu gehn,\\*
standen sie in entlegener Au\\
bei einem blühenden Strauch,\\*
und er suchte die Hände der Frau,

und sie senkte den Kopf tief, als er sang:\\*
Dein Danklied soll erschallen\\
[s 105]
durch alle Wälder und über das Meer\\*
und in hochgebauten Hallen.

Und als sie fortgegangen war,\\*
verlor sich das Lied wie ein Traum.\\
Und er lehnte nur in den Nächten stumm\\*
an einem alten Baum.
<v end>

[. . |]Der Sterbende wandte den Kopf, daß der
volle Schein der Kerzen darauf fiel.

„Laß es gehen.“

Er senkte den Kopf wieder in den Schatten des
Kissens zurück.

„Wie alt sie geworden sind,“ sagte er.

Die Stille des Gemaches wurde nimmer durch­
brochen. Der Regen, der auf die Fenster fiel, verwob
sich damit.

Und eines Nachts lag er unter den flackernden
Kerzen gestreckt wie in Lust, gestorben<l vater_ende>[ . . .
|.
<aa>]
Hans Jung ging den Weg durch die hohen Bäume
zurück. Ein leiser Regen war gekommen, ohne daß
er es gemerkt hatte, und der Wind hatte aufgehört,
durch die Blätter zu gehen. Es war still, daß er
seine Schritte in dem Wald verklingen hörte.

Wo der Weg aus dem Wald in die Wiesen
mündete, blieb er stehen. Seine große Gestalt stand
einsam in dem Bogen der Bäume, die den Wald schlossen.

[s 106]
Kein Licht brannte mehr im Dorf unten. Die
Häuser waren zu einer schwarzen Masse zusammen­
geflossen. Der See leuchtete nur matt aus dem Dunkel
heraus.

Hans Jung wandte sich, in sein Zimmer hinab­
zugehen.

Aber ehe er ging, lachte er auf einmal auf, grell
in die Stille, sinnlos.

[s 107]

